\documentclass[a4paper,11pt]{article}

\usepackage[utf8]{inputenc}
\usepackage{url}
\usepackage{hyperref}
\usepackage{authblk}

\title{Contract Orchestration for OCaml \\
  Progress Report 
}
\author{Thi Thu Ha Doan}
\author{Peter Thiemann}
\affil{University of Freiburg, Germany}
\begin{document}
\maketitle{}

\section{Introduction}
\label{sec:introduction}

The Contract Orchestration for OCaml (COOC) project is funded by the Tezos foundation. Its overall goal  is to provide a framework that guarantees the correct orchestration of contract invocations from an application program in the OCaml language. This report is a continuation of the first report dated July 31, 2020.

\subsection{Personnel}
\label{sec:personnel}

There is one full-time employee working on the project for the whole time period: Dr. Thi Thu Ha Doan (postdoctoral researcher). 

Student workers (part time as explained below): Tamara Bernhardt, Joshua Allgeier

\section{Timeline of Events}
\label{sec:timeline-events}

\paragraph{August 2020 - October 2020} 
\begin{itemize}
\item
Since May 2020, Tamara Bernhardt, Master's student, has been employed as a student woker. She designed and implemented the Ocaml API that interacts with the Tezos node. The Ocaml API is based on the intensive use of the recently released Tezos library on Opam. The research  also aims at exploring the library intensively. The Ocaml API provides a programmatic interface that allows users to interact with the Tezos node, such as transferring tokens, invoking and initialising a smart contract, and querying the state of an operation. It leverages the Tezos client library with its built-in validation and pre-validation of operations to provide immediate feedback and avoid injections of invalid operations. In addition, the API extends these functions to handle the fee. A manager operation is issued starting with a low fee and repeated with the higher fee if it is rejected by the baker due to the low fee. It also handles return errors structurally. The implementation of the API can be found at the Github repository \cite{tezos-api}.

\end{itemize}
An application may perform some offline calculations before passing the result as a parameter to a smart contract on the blockchain to reduce the cost of gas and fees. However, the parameters should be checked for legitimacy, i.e., they should satisfy some specific properties. So we need a mechanism to check the validity of the assumptions before the contract is executed. It is not trivial to confirm these properties, as they  require a large amount of computation, often depending on the size of the parameter. Therefore, our goal is to design and implement a checking assertion that verifies the legitimacy of a smart contract parameter. The research proposal can be found at the expos\'{e} by Peter Thiemann \cite{expose}.

Consider a property that can be expressed with an explicit assertion in predicate logic. If we test the validity of a parameter within a smart contract, the test would consume additional cost according to the size of the parameter. However, we could do better by recruiting the validators of the contract for a distributed attempt to find a counterexample. To this end, we consider the negation of an assertion. The assertion can be checked by having each validator work independently. In this scenario, each validator only needs to be paid to generate a random value, say a random number, and perform a computation, which is a constant cost that is independent of the size of the parameter. To this end, our strategy is to extend the contract language with assertions that specify these logical formulas.

\paragraph{October 2020 - June 2021}
\begin{itemize}
\item Tamara Bernhardt worked her master's thesis (thesis work is generally unpaid, but supervised by Dr.\ Doan) on distributed  assertion checking for Tezos smart contracts. The assertion is specified in its own syntax in a separate file from the contract implementation. This scenario requires a separate compiler for assertions.  The generated code is either combined with the contract implementation or stored separately on the blockchain. Tamara worked on grammar, lexer/parser, transformer, and the compiler for the assertion language. 
\begin{itemize}
\item Defining the grammar for the assertion language. Although the underlying type structure is inspired by Michelson, it was designed to be as general as possible so that it can be adapted to different smart contract languages. As Tamara did not have  experience with parsing, she created a parser for a Scheme-style prefix syntax using Menhir. Another student was scheduled  to implement an alternative grammar inspired by OCaml syntax (as a term project), but he did not finish the work.
\item Implementing and extensively testing the lexer/parser using OCamllex and Menhir for the prefix style version.
\item Writing a backend transformer for a specific target. The backend interface is built as a virtual library for which multiple  implementations can be provided.
\item Implementing the compiler that generates the target code according to an on-chain orchestration scheme between the assertion and the parent contract.
\end{itemize}
At the end, Tamara wrote the development document and then wrapped it up in her thesis.
\end{itemize}

\paragraph{November 2020 - June 2021}
\begin{itemize}
\item We supervised the master's thesis of Julian Veigel. He developed a counterpart of the assertion checker for Ethereum. He  and Tamara collaborated on some parts of the assertion project, e.g., on the assertion language and the general analysis of the problem (in particular, stastistical and economic aspects). Julian completed his thesis in mid-June. It deals with different approaches on how to implement such a validation process. The basic idea is to distribute the validation process to several nodes in the network, all testing different property cases. The goal is to reduce the cost while preserving basic properties such as decentralization and trust. All the approaches are discussed in detail and the most promising one is shown in an implementation. The contribution of his work includes:
\begin{itemize}
\item exploring Ethereum infrastructures, especially the recent release of Ethereum 2.0.
\item presenting different approaches and discuss their strengths and weaknesses.
\item laying some mathematical groundwork to build a qualitative.
\item an implementation with its components for one of the presented approaches.
\end{itemize}
\end{itemize}

Julian's work is not directly related to Tezos. Nevertheless, it contributed to the project goals because we gained a lot of insight into the problem due to the different perspective provided by considering Ethereum. Julian initiated and pursued the statistical analysis as well as the investigation of the economic aspects as part of his master's thesis.


\paragraph{April 2021}
\begin{itemize}
\item Leqaa Ahmed, a bachelor student in computer science, starts her bachelor thesis in my group. Her task is to further develop and strengthen the typing properties of our Tezos-API. She is mainly supervised by Dr. Doan.
\end{itemize}

Dr. Doan (mainly) and I have been running the course “Blockchain and Cryptocurrencies”. This course builds on the book “Bitcoin and Cryptocurrency Technologies” by Narayanan and others. Our implementation of the course covers seven weeks worth of material from the book; two weeks on Ethereum, Solidity, and its security issues; and two weeks on Tezos, with the bulk of material on Michelson. 
\paragraph{May 2020 - September 2020}
\begin{itemize}
\item First instance of the course with 25 students. We created the lecture materials consisting of slides and videos.  We designed numerous exercise tasks to help students confirm their understanding of the basic concepts of cryptocurrencies and blockchains. We also focus on training students in blockchain ``smart contract'' programming. Students are required to implement some basic blockchain systems in  Python. The they write smart contracts in Solidity and Michelson. In terms of Michelson, we guided the students to understand how the languages work and study its strong type system. For practice, we use some online tools that support coding and testing of smart contracts in Solidity (Remix ethereum) and Michelson (Try michelson). Finally, we conducted the final exam in the form of a take home exam during a time window of 48 hours. The final exam covers the main concepts and several programming tasks. The grading of the exam was completed and submitted at the end of October. 
\end{itemize}

\paragraph{April 2021 - September 2021}
\begin{itemize} 
\item Second instance of the course with 39 students. The basic content of the lecture and exercises has remained essentially the same, but has  been adjusted according to feedback we have received from former students. Dr.\ Doan is conducting the weekly exercise sessions while lectures are mainly provided with recordings from the first instance. We plan to create one new lecture to be delivered in July. The exam will be in September in the same format as in the first instance. 
\end{itemize}



Doan and Thiemann have worked on the safe interaction with contracts on the Tezos blockchain.
\paragraph{August 2020  - June 2021}
\begin{itemize} 
\item Tezos provides low-level interfaces, which work with loosely structured data in JSON format that offers few static guarantees. We proposed Contract Modules that provide a statically checked interface to interact with contracts on the Tezos blockchain. A module specification provides all types as well as information about potential failure conditions of the contract. The specification is checked against the contract implementation using symbolic execution. An OCaml module is generated that contains a function for each entrypoint of the contract. The types of these functions fully describe the interface including the failure conditions and guarantee type-safe and sometimes fail-safe invocation of the contract on the blockchain.

Based on the result of the research, we submitted a short paper titled ``Towards Contract Modules for the Tezos Blockchain''  to the workshop on Formal Methods for Blockchains (FMBC 2021). 

\item Current blockchains mostly provide RPC interfaces, such as the Ethereum JSON-RPC API and the Tezos RPC API, but they require cumbersome manipulation of string data in JSON format and do not provide static
guarantees (except that the response to a well-formed JSON input is also a well-formed JSON output). To improve on this situation, we propose a typed API for for originating and invoking contracts as well as observing the state of the blockchain and establish some properties of the combined system. Specifically, we provide an execution model that enables us to prove type-safe interaction between programs and the blockchain. We establish further properties of the model that give rise to requirements on the API. A prototype of the interface is implemented in OCaml for the Tezos blockchain.  Our typed API supports the
implementation of application programs and oracles that safely interact with
smart contracts on the blockchain. Moreover, our approach provides a type-safe
facility to communicate with contracts where data is automatically marshalled
between OCaml and the blockchain. This interface is a step towards a seamless integration of contracts into traditional programs.

As a result, we submitted a research paper to the Asian Symposium on Programming Languages and Systems (APLAS 2021). The contributions of the paper are as follows:
\begin{itemize}
\item A typed API for originating and invoking contracts as well as querying the
state of the blockchain.
\item An operational semantics for functional programs running alongside smart
contracts in a blockchain.
\item Established various properties of the combined system.
\item An implementation of a low-level OCaml-API to the Tezos blockchain, which corresponds to the operational semantics. 
\end{itemize}
\end{itemize}

\section{Relation to the Contract}
\label{sec:relation-contract}

We have made progress on several points in the contract.
\begin{itemize}
\item Type system of OCaml and Michelson to ensure smooth interaction: this work is in progress in the submitted papers. Final steps to be taken in the work of Ms Leqaa Ahmed.
\item Framework to guarantee orchestration
  \begin{itemize}
  \item First draft of contract modules provided in paper submitted to FMBC.
  \item Compiler for contract modules: this is our next working item.
  \item Behavioral type system:
    We created a preliminary symbolic interpreter for checking contract module specifications. We plan to prototype the behavioral type system by extending the specifications with pre- and postconditions, as well as constraints on the possible call orders of the entrypoints. This approach gives us sufficient flexibility to investigate different designs using an encoding in first-order logic and using an SMT solver as a type checker for the system. 
  \end{itemize}
\item (Linear types and typestate have not made it into OCaml so far, but)
  \begin{itemize}
  \item type-safe integration of Tezos-RPC calls in OCaml: see the paper submitted to APLAS
  \item behavioral types with resource: this will be integrated in the SMT-based approach outlined above
  \item type checker: will only make sense after evaluating the design space as explained above
  \end{itemize}
\item Mini-OCaml with operational semantics: see APLAS submission
\item We are in the second year of teaching blockchain concepts (see above) and start getting traction with students (two master's theses finished, and currently  several Bachelor students on the topic). We are also offering an intensive course in OCaml to attract students to work on Tezos and project related tasks.
\item We gave a presentation based on the current state of the project as part of the Nomadic Lab seminar series.
\end{itemize}

\section{Next Steps}

We plan to 
\begin{itemize}
\item continue our teaching activities (the blockchain course, a seminar on the topic in the winter semester, another instance of the intensive OCaml course);
\item implement code generation for contract modules;
\item extend contract modules to contract specifications that include pre- and postconditions as well as permissible call sequences for entrypoints; 
\item design and implement a checker for contract specifications based on Michelsym, our prototype symbolic interpreter for Michelson;
\item complete the implementation of the type-safe interface (Leqaa Ahmed);
\item finish the implementation of assertions (cf.\ the theses of Tamara Bernhardt and Julian Veigel), document the outcome in a paper.
\end{itemize}
We also plan to redesign the redesign the OCaml interface based on concepts from concurrent programming (promises / futures).



\bibliography{report}
\bibliographystyle{plainurl}
\end{document}
