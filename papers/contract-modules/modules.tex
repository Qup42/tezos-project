\documentclass[a4paper]{llncs}
\usepackage[english]{babel}
\usepackage[utf8]{inputenc}
\usepackage{amsmath}
\usepackage{amssymb}
\usepackage{graphicx}
\usepackage[colorinlistoftodos]{todonotes}
\usepackage{mathpartir}
\usepackage{fixme}
\usepackage{listings}

\lstset{
  language=caml,
  morekeywords={contract,entrypoint,module,hash,paid,raises,sig,val},
  % predefined types
  classoffset=1,
  morekeywords={nat,int,unit,string,pukh,status,tz},
  keywordstyle=\color{blue},
  classoffset=2,
  morekeywords={Left,Right},
  keywordstyle=\color{magenta},
  backgroundcolor=\color{brown!20},
  captionpos=b,
  literate={->}{{$\to$}}1
}

\lstdefinelanguage{michelson}{
  % basicstyle=\ttfamily{},
  classoffset=0,
  morekeywords={parameter,storage,code},
  keywordstyle=\bfseries, 
  % predefined types
  classoffset=1,
  morekeywords={nat,int,unit,string,or,pair,option,address},
  keywordstyle=\color{blue},
  classoffset=2,
  morekeywords={CAR,CDR,CONS,DUP,EQ,FAILWITH,IF,IF_LEFT,LE,LEFT,NIL,DROP,PUSH,RIGHT,SENDER,SUB},
  keywordstyle=\color{magenta},
  backgroundcolor=\color{pink!20}
}

\title{Whitepaper: Contract Modules}
\author{Peter Thiemann}
\institute{}

\date{\today}

\begin{document}
\maketitle
\pagestyle{plain}

\section{Motivation}
\label{sec:motivation}


Consider this Michelson code fragment:
\begin{lstlisting}[language=michelson,caption={Michelson code example},label={lst:code-example}]
parameter (or (int %bid_for_auction)
              (unit %close_auction));
storage address;
code { DUP; CAR; 
       IF_LEFT {
       # %bid_for_auction
        DUP; PUSH int 10000; SUB; LE;
        IF { FAILWITH; } {} # fails with argument value
        # ...
        { PUSH string "closed"; FAILWITH; }
        # ...
        { PUSH string "bid too low"; FAILWITH; }
        # ...
        { PUSH int 42; FAILWITH; }
        # ...
       }
       {
       # %close_auction
         DROP; CDR; SENDER; EQ;
         IF {} { PUSH string "not owner"; FAILWITH; }
         # ...
       }
     }
\end{lstlisting}

Here is a contract type that describes / specifies the above Michelson code.
\begin{lstlisting}[caption={Contract module example},label={lst:contract-module-example}]
contract type MyContract = sig
  paid entrypoint %bid_for_auction (value : int) 
  raises "closed"                 (** auction closed*)
       | "bid too low" as too_low (** bid received is too low *)
       | 42 as the_answer         (** failwith received argument 42 *)
       | (_ : int) as failedwith  (** failwith received an integer arg *)

  entrypoint %close_auction (_ : unit)
  raises "not owner"         (** caller cannot close the auction *)
end
\end{lstlisting}

The body of the \lstinline/sig ... end/ can be stored in a
\texttt{.tzm} file where the basename matches the one of the
\texttt{.tz} file. The output is a \texttt{.mli} file which contains a
valid OCaml signature. In the next step, we are also generating the
corresponding \texttt{.ml} file implementing the signature.

Each \lstinline/entrypoint/ is mapped to a function. The final pattern
variable may be a wildcard \lstinline/_/ or an OCaml identifier, which
is used in generating the implementation. For \lstinline/_/ a standard
name will be generated.

A \lstinline/paid/ entrypoint requires the caller to specify an amount of tokens to
accompany the contract invocation. Accordingly, the generated function has an extra
argument. Otherwise, the entrypoint is unpaid. 

The \lstinline/raises/ clauses specify values that the contract might
pass to the \texttt{FAILWITH} instruction. The \lstinline/as/ clause
maps the matched value to an OCaml identifier, which is used to
generate the corresponding case in the error type. There is a standard
translation (to be specified) that maps a value without an
\lstinline/as/ clause to an identifier.

Alternatively and equivalently, the contracts can be specified using
patterns on the default entrypoint.

\begin{lstlisting}[caption={Alternative syntax for entrypoints},label={lst:alternative-syntax}]
  paid entrypoint bid_for_auction (Left (_ : int))
...
  entrypoint close_auction (Right (_ : unit))
\end{lstlisting}

The contract module is compiled to an OCaml module which provides a typed API to the
contract. It includes the documentation comments from the contract
module. Internally, the typed API builds on the untyped API developed
by us. 

\begin{lstlisting}[caption={Generated signature},label={lst:generated-signature}]
module type MyContract =
sig
  type bid_for_auction_errors = 
     | closed       (** auction closed*)
     | too_low      (** bid received is too low *)
     | the_answer   (** failwith received argument 42 *)
     | failedwith of int (** failwith received an integer arg *)

  val bid_for_auction
    : Tezos.pukh -> ?fee:Tezos.tz -> Tezos.tz
    -> int -> (Tezos.status, bid_for_auction_errors) Tezos.monad

  type close_auction_errors = 
     | not_owner    (** caller cannot close the auction *)

  val close_auction
    : Tezos.pukh -> ?fee:Tezos.tz -> ?amount:Tezos.tz
    -> unit -> (Tezos.status, close_auction_errors) Tezos.monad
end
\end{lstlisting}

\begin{itemize}
\item The first argument is the public key hash of the caller. 
\item The \lstinline/paid/ entrypoint requires an \lstinline/amount/
  argument, which is optional (with default 0) for an unpaid
  entrypoint.
\item The \lstinline/fee/ parameter is optional in both
  cases. It is precomputed and supplied as the default value in the
  implementation.

  Q (extension): Would it make sense to specify something about fees in the
  contract module? For example, we might specify a fee schedule which
  raises the fee until the call is accepted.
\item The type \lstinline/Tezos.monad/ is defined by the underlying
  API. The untyped API uses a type \lstinline/Answer.t/. The exact
  relationship is TBD.
\item Error reporting assumes that the monad returns a sum type. But
  things could be handled differently (for the
  \lstinline/close_auction/entrypoint): by an error continuation of
  type \lstinline/close_auction_errors -> Answer.t/ or by
  raising an exception that encapsulates the error type:
\begin{lstlisting}
exception Close_Auction_Exn of close_auction_errors
\end{lstlisting}
\end{itemize}

If the contract source is available, then an extended type checker verifies that the entry points
exist and that all failures that may arise during contract execution from a particular entry point
are covered by the listed cases.

If the contract source is unavailable, then we can obtain it from the
contract's public hash using an API call. In that case, we would
specify the hash in the contract.
\begin{lstlisting}
sig
  hash "KT1ThEdxfUcWUwqsdergy..."
...
\end{lstlisting}

We propose to treat the remaining errors (i.e., built-in errors
and failures that are not specifed) like Java's unchecked
exceptions. That is, the API function throws an exception.


\begin{lstlisting}[caption={Generated implementation},label={lst:generated-implementation}]
(* constant template *)
open Tezos_api
open Tezos_serialization (* Tint, Tunit, serialize, ... *)

let failwith_regex = 
  "A FAILWITH instruction was reached.*"
let int_regex = 
  "[+-]?[0-9]+"

(* variable part *)
let contract_addr = 
  get_contract "KT1ThEdxfUcWUwqsdergy..." >>=? fun contract -> ???

type bid_for_auction_errors = 
   | closed       (** auction closed*)
   | too_low      (** bid received is too low *)
   | the_answer   (** failwith received argument 42 *)
   | failedwith of int (** failwith received an integer arg *)

let big_for_auction_errors_map =
  let open Base in
  Base.Map.of_alist_exn (module String)
    [
      (failwith_regex ^ "closed", closed);
      (failwith_regex ^ "bid too low", too_low);
      (failwith_regex ^ "42", the_answer);
      (failwith_regex ^ int_regex, failedwith (0)); (* badness 10000 *)
    ]
  (* inadequate in several ways;
   * should just generate code instead of using a map*)
  

let bid_for_auction destination ?(fee=100.0) amount value =
  ...
  begin
    SyncAPIV0.call_contract
      amount
      destination
      contract_addr
      ~(serialize Tint value)
      fee
    >>=? fun oph ->
    wait_for_inclusion oph (* busy waiting *)
    >>=? fun _ ->
    return 0
  end
  >>= function
  | Ok _ -> ...
  | Error (Rejection Michelson_runtime_error s) -> ...

type close_auction_errors = 
   | not_owner    (** caller cannot close the auction *)

let close_auction destination ?(fee=50.0) ?(amount=0.0) _ =
  ...
  SyncAPIV0.call_contract
    amount
    destination
    contract_addr
    ~(serialize Tunit)
    fee
\end{lstlisting}


\clearpage
\section{Tools}
\label{sec:tools}

An extended type checker collects information about explicit failures along execution paths
starting from each possible entry point. This checker will be able to
verify the contract module specification in
\ref{lst:contract-module-example} against the Michelson code in
\ref{lst:code-example}. We may also wish to provide an entry point so
that the checker only gathers information for that entry point. This alternative may be interesting if there
is an entry point taking a parameter of sum type, rather than using all injections to distinguish
between entry points.

The checker should warn about unhandled failures (both checked and unchecked).

The recommended technology for the type checker is symbolic
execution. In this framework, it would also 
be possible to specify further invariants about the inputs and place them into the contract
module. Essentially, every formula that an SMT solver can deal with
should be fine, although a first version of the symbolic executor
could work without using SMT.

The checker should also make sure that unpaid entrypoints do not
depend on a non-zero amount being passed to the contract
invocation. For example, the \texttt{AMOUNT} instruction of Michelson
should not be used in the code.  

The checker should precalculate the fee. It essentially obtains
this value from the Tezos library.

Failures might arise in contracts that are transitively invoked. In
this case, the checker would obtain the callees' code and process
it. The first step, not too difficult, in this direction would be to
approximate the contents of the list of operations that is produced by
the contract. 

A starting point for the implementation could be the module \lstinline/Tezos_micheline/, which has a
Michelson parser. 

\end{document}
