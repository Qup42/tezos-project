\documentclass[a4paper]{llncs}
\usepackage[english]{babel}
\usepackage[utf8]{inputenc}
\usepackage{amsmath}
\usepackage{amssymb}
\usepackage{graphicx}
\usepackage[colorinlistoftodos]{todonotes}
\usepackage{mathpartir}
\usepackage{fixme}
\usepackage{listings}

\lstset{language=caml}

\title{Contract Modules}
\author{}
\institute{}

\date{\today}

\begin{document}
\maketitle
\pagestyle{plain}

\section{Motivation}
\label{sec:motivation}



\begin{lstlisting}[language=caml]
contract module MyContract = "kt1..."
  entry point bid_for_auction Left (_ : int) 
  raises "closed" as closed       (* auction closed*)
         "bid too low" as too_low (* bid received is too low *)
         42 as the_answer         (* failwith received argument 42 *)
         (_ : int) as failedwith  (* failwith received an integer arg *)

  entry point close_auction Right (_ : unit)
  raises "not owner"         (* caller cannot close the auction *)
\end{lstlisting}

The contract module is compiled into an OCaml module which provides a typed API to the
contract. Internally, the typed API builds on the untyped API developed by us.

\begin{lstlisting}[language=caml]
signature MyContract =
sig 
  type bid_for_auction_errors = 
     | closed 
     | too_low 
     | the_answer
     | failedwith of int

  val bid_for_auction : int -> x Tezos

  type close_auction_errors = 
     | not_owner

  val close_auction : unit -> x Tezos
end
\end{lstlisting}

If the contract source is available, then an extended type checker verifies that the entry points
exist and that all failures that may arise during contract execution from a particular entry point
are covered by the listed cases.

Q: how should we deal with built-in errors? Like Java's (unchecked?) exceptions?

If only the public hash of the contract is known, then obtain the source code through the tezos
client and proceed as above.

\clearpage
\section{Tools}
\label{sec:tools}

An extended type checker that collects information about explicit failures along execution paths
starting from each possible entry point. We may also wish to provide an entry point so that the
checker only gathers information for that entry point. This alternative may be interesting if there
is an entry point taking a parameter of sum type, rather than using all injections to distinguish
between entry points.

It seems like the type checking is doable using symbolic execution. In this framework, it would also
be possible to assume further invariants about the inputs and place them into the contract
module. Essentially, every formula that an SMT solver can deal with should be fine.

A starting point for the implementation could be the module \lstinline/Tezos_micheline/, which has a
Michelson parser. 

\end{document}
