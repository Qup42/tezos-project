\documentclass[a4paper,USenglish,american,cleveref, autoref, thm-restate]{oasics-v2021}
%This is a template for producing OASIcs articles. 
%See oasics-v2021-authors-guidelines.pdf for further information.
%for A4 paper format use option "a4paper", for US-letter use option "letterpaper"
%for british hyphenation rules use option "UKenglish", for american hyphenation rules use option "USenglish"
%for section-numbered lemmas etc., use "numberwithinsect"
%for enabling cleveref support, use "cleveref"
%for enabling autoref support, use "autoref"
%for anonymousing the authors (e.g. for double-blind review), add "anonymous"
%for enabling thm-restate support, use "thm-restate"
%for enabling a two-column layout for the author/affilation part (only applicable for > 6 authors), use "authorcolumns"
%for producing a PDF according the PDF/A standard, add "pdfa"

\listfiles

%\graphicspath{{./graphics/}}%helpful if your graphic files are in another directory

\bibliographystyle{plainurl}% the mandatory bibstyle

\title{Towards Contract Modules for the Tezos Blockchain} %TODO Please add

% \titlerunning{Dummy short title} %TODO optional, please use if title is longer than one line

\author{Thi Thu Ha Doan}{University of Freiburg,
  Germany}{doanha@cs.uni-freiburg.de}{https://orcid.org/0000-0002-1825-0097}{supported
by a grant from the Tezos foundation}%TODO mandatory, please use full name; only 1 author per \author macro; first two parameters are mandatory, other parameters can be empty. Please provide at least the name of the affiliation and the country. The full address is optional

\author{Peter Thiemann}{University of Freiburg, Germany}{thiemann@acm.org}{https://orcid.org/0000-0002-9000-1239}{}

\authorrunning{T.T.H Doan and P. Thiemann} %TODO mandatory. First: Use abbreviated first/middle names. Second (only in severe cases): Use first author plus 'et al.'

\Copyright{Thi Thu Ha Doan and Peter Thiemann} %TODO mandatory, please use full first names. OASIcs license is "CC-BY";  http://creativecommons.org/licenses/by/3.0/

%\ccsdesc[100]{\textcolor{red}{Replace ccsdesc macro with valid one}}
%%TODO mandatory: Please choose ACM 2012 classifications from
%%https://dl.acm.org/ccs/ccs_flat.cfm
\ccsdesc[100]{}

\keywords{contract API, modules, static checking} %TODO mandatory; please add comma-separated list of keywords

\category{} %optional, e.g. invited paper

\relatedversion{} %optional, e.g. full version hosted on arXiv, HAL, or other respository/website
%\relatedversiondetails[linktext={opt. text shown instead of the URL}, cite=DBLP:books/mk/GrayR93]{Classification (e.g. Full Version, Extended Version, Previous Version}{URL to related version} %linktext and cite are optional

%\supplement{}%optional, e.g. related research data, source code, ... hosted on a repository like zenodo, figshare, GitHub, ...
%\supplementdetails[linktext={opt. text shown instead of the URL}, cite=DBLP:books/mk/GrayR93, subcategory={Description, Subcategory}, swhid={Software Heritage Identifier}]{General Classification (e.g. Software, Dataset, Model, ...)}{URL to related version} %linktext, cite, and subcategory are optional

%\funding{(Optional) general funding statement \dots}%optional, to capture a funding statement, which applies to all authors. Please enter author specific funding statements as fifth argument of the \author macro.

% \acknowledgements{I want to thank \dots}%optional

%\nolinenumbers %uncomment to disable line numbering

%\hideOASIcs %uncomment to remove references to OASIcs series (logo, DOI, ...), e.g. when preparing a pre-final version to be uploaded to arXiv or another public repository

%Editor-only macros:: begin (do not touch as author)%%%%%%%%%%%%%%%%%%%%%%%%%%%%%%%%%%
\EventEditors{John Q. Open and Joan R. Access}
\EventNoEds{2}
\EventLongTitle{42nd Conference on Very Important Topics (CVIT 2016)}
\EventShortTitle{CVIT 2016}
\EventAcronym{CVIT}
\EventYear{2016}
\EventDate{December 24--27, 2016}
\EventLocation{Little Whinging, United Kingdom}
\EventLogo{}
\SeriesVolume{42}
\ArticleNo{23}
%%%%%%%%%%%%%%%%%%%%%%%%%%%%%%%%%%%%%%%%%%%%%%%%%%%%%%

%% structure
\newcommand{\Angle}[1]{\langle#1\rangle}

%% values
\newcommand\SUNIT{\textbf{()}}
\newcommand{\TRUE}{\textbf{True}}
\newcommand{\FALSE}{\textbf{False}}

%% names
\newcommand{\ALS}{\textbf{als}}
\newcommand{\PAK}{\textbf{pak}}
\newcommand{\PUK}{\textbf{puk}}
\newcommand{\PKH}{\textbf{pkh}}
\newcommand{\PUH}{\textbf{puh}}
\newcommand{\CODE}{\textbf{code}}
\newcommand{\BAL}{\textbf{bal}}
\newcommand{\COU}{\textbf{cou}}
\newcommand{\STORAGE}{\textbf{storage}}
\newcommand{\OP}{\textbf{op}}
\newcommand{\OPH}{\textbf{oph}}
\newcommand{\TIME}{\textbf{t}}
\newcommand{\CONTRACTORS}{\textbf{T}}
\newcommand{\PENDING}{\textbf{P}}
\newcommand{\ACCEPTED}{\textbf{A}}
\newcommand{\MANAGERS}{\textbf{K}}
%% operations
\newcommand{\TRANSFER}[5][\SUNIT]{\text{transfer $#2$ from $#3$ to $#4$ arg $#1$ fee $#5$}}
\newcommand{\ORIGINATE}[6]{\text{originate contract $#1$ transferring $#2$ from $#3$ running $#4$ init $#5$ fee $#6$}}
\newcommand{\NTEZ}{\textbf{n}}
\newcommand{\MTEZ}{\textbf{m}}
\newcommand{\ID}{\textbf{id}}
\newcommand\STRING{\textbf{s}}
%% queries
\newcommand{\QRY}{\textbf{qry}}
\newcommand{\GETBALANCE}[1]{\text{get balance for $#1$}}
\newcommand{\GETSTATUS}[1]{\text{get status for $#1$}}
\newcommand{\GETSTORAGE}[1]{\text{get contract storage $#1$}}
\newcommand{\GETCODE}[1]{\text{get code for $#1$}}
\newcommand{\GETPUBLICKEY}[1]{\text{get public key for $#1$}}
\newcommand{\GETCOUNTER}[1]{\text{get counter for $#1$}}

\newcommand{\ACCOUNTS}{\textbf{C}}
\newcommand{\OPERATIONS}{\textbf{O}}
\newcommand{\CONTRACTS}{\textbf{S}}

\newcommand{\NODE}{\textbf{N}}
\newcommand{\BLOCKCHAIN}{\textbf{B}}

%% functions
\newcommand{\CHECKACC}{\textup{checkAcc}}
\newcommand{\CHECKID}{\textup{checkId}}
\newcommand{\CHECKBAL}{\textup{checkBal}}
\newcommand{\CHECKCOU}{\textup{checkCou}}
\newcommand{\CHECKPUB}{\textup{checkPub}}
\newcommand{\UPDATECOU}{\textup{updateCou}}

\newcommand{\GENERATEOPH}{\textup{generateOph}}

%% transition relations
\newcommand{\NodeTrans}{\longrightarrow_N}
\newcommand{\SystemTrans}{\longrightarrow}


\begin{document}

\maketitle

%TODO mandatory: add short abstract of the document
\begin{abstract}
  Programmatic interaction with a blockchain is often clumsy.
  Many interfaces handle only loosely structured data, often in JSON
  format, that is inconvenient to handle and offers few guarantees.

  Contract modules provide a statically checked interface to interact
  with contracts on the Tezos blockchain. A module specification
  provides all types as well as information about potential failure
  conditions of the contract. The specification is checked against the
  contract implementation using symbolic execution. An OCaml module is
  generated that contains a function for each entrypoint of the
  contract. The types of these functions fully describe the interface
  including the failure conditions and guarantee type-safe and
  sometimes fail-safe invocation of the contract on the blockchain.
\end{abstract}

\section{Introduction}
\label{sec:introduction}

Contracts on the blockchain rarely run in isolation. To be useful
beyond shuffling tokens between user accounts, they need to interact
with the outside world. On the other hand, the outside world also
needs to interact by initiating transactions and starting contracts
that feed information into the blockchain. One direction is addressed
by oracles that watch certain events on the blockchain, create a
response by calculation or gathering data, and then invoke a callback
contract to inject this response into the chain. Trust is an essential
aspect for an oracle.

The other direction is about automatizing certain processes in
connection with the blockchain. For example, opening or closing an
auction according to a schedule, programming a strategy for an
auction, or creating an NFT. To this end, an interface is needed to
invoke contracts safely. Existing interfaces are lacking because they
are essentially untyped (string-based or JSON-based) and often low
level because they require dealing directly with RPC interfaces. Trust
is not needed because the process runs on behalf of a certain user.

Contract modules provide a clean, language-integrated way to interact
with a blockchain. They abstract over underlying string-based
interfaces and details like fee handling. They provide a high-level
typed interface which reduces a contract invocation to a function call
in the language.

Contract modules do not provide a fixed API, but
rather generate a specialized interface for each contract. This
interface is statically checked against the contract implementation to
ensure type safety and exception safety (every failure condition
arising is handled by proper error reporting).

Our work is situated in the context of the Tezos blockchain, which
supports Michelson as its low-level contract language, and the
language OCaml, which comes with an expressive polymorphic type system
as well as a powerful module system that we
enhance with contract modules. 

\section{Context}
\label{sec:context}

Tezos is a third generation, account-based, self amendable
blockchain \cite{tezos-whitepaper}. It employs a proof-of-stake consensus protocol, which
includes ways to evolve the protocol itself. The consensus protocol is
executed by so-called bakers and their proposed blocks are checked by
validators. They receive some compensation in the form of
tokens (Tezzies) for their work. According to proof-of-stake,
bakers and validators are just nodes elected by the Tezos network
according to their token balance. 

Each Tezos contract owns an account as well as some storage. Contracts
are pure functions of type parameter $\times$ storage $\to$ operation
list $\times$ storage. When a contract is invoked with a parameter,
the blockchain provides the current storage and updates it with the
second, storage component of its return value. The first component is
a list of blockchain operations (contract deployments, token
transfers, contract invocations) that are executed transactionally
after the first invocation terminates. Each invocation may be
accompanied with an amount of tokens that are added to the current
account balance of the contract.

Contracts are implemented in the language Michelson, a fully typed
stack-based language. Each contract has fixed types for its parameter
and for its storage. The storage is initialized when the contract is
deployed. Besides primitive types like unit, int, bool, address, and string,
there are pairs, sums, functions, lists, and maps (and many more) that
can serve as types for storage and parameters.

\section{An auction contract}
\label{sec:an-auction-contract}
\begin{lstlisting}[float,caption={Simple auction contract (auction.tz)},captionpos=b,label={lst:auction-header},language=michelson,numbers=none,emph={close,bid},emphstyle=\underbar]
parameter (or (unit %close) (unit %bid));
storage (pair bool          # bidding allowed
         (pair address      #  contract owner
          address           # highest bidder's address
        ));
\end{lstlisting}

As a concrete example, we consider a simple auction contract with the
header shown in Listing~\ref{lst:auction-header}. 
This contract has two entrypoints, \lstinline/close/ and \lstinline/bid/,
expressed by giving the single parameter a sum type. To call the
entrypoint \lstinline/close/ we invoke the contract with parameter \lstinline/Left ()/
otherwise we use \lstinline/Right ()/, where \lstinline/()/ is the sole
value of type \lstinline/unit/. The contract's storage is a nested pair which
contains a boolean flag and two addresses.

The contract works as follows. It is deployed with storage
\lstinline/(true,(owner,owner))/ which indicates that bidding is
allowed and the contract owner is currently also the highest
bidder. On deployment the owner also deposits an initial balance to indicate the
minimum bid. Closing the contract transfers the balance to the
owner. It is restricted to the owner. Bidding fails if the contract is
closed. If bidding is open and the amount of tokens with the bid exceeds the
current highest bid, the current bidder replaces the previous highest
bidder and the previous highest bidder is reimbursed. Otherwise,
bidding fails, too.

To invoke this contract from an OCaml program, we'd like to generate an
OCaml module, say \lstinline/Auction/, from a specification of the
contract. This module contains two functions
\lstinline/close/ and \lstinline/bid/ corresponding to the
entrypoints. The type of these entrypoints reflects further properties
of these entrypoints as well as the ways in which an entrypoint might
fail.

Besides the obvious, technology induced ways that a contract invocation might fail
(insufficient gas price offered, insufficient gas to complete, timeout
due to lack of connectivity, etc) a Michelson contract can fail due to
a programmer induced condition. To this end, there is a Michelson
instruction \lstinline/FAILWITH/ that terminates contract execution
with an error message which is reported back to the caller. This error
message includes a rendering of the top value on the stack.

We consider the technological failures like Java's unchecked
exceptions, but we wish to deal with the explicit failures like
checked exceptions.
Our generated code handles failures in a suitable
error monad that makes the failures explicit in a custom
datatype.\footnote{Alternatively, this could be done using OCaml
  exceptions, but we chose to stay within the monadic framework that
  is already used by existing Tezos APIs.}

\begin{lstlisting}[float,captionpos=b,caption={Contract module example},label={lst:contract-module-example}]
contract type Auction = sig
  paid entrypoint bid () 
  raises "closed"               (** auction closed *)
       | "too low"              (** bid too low    *)

  entrypoint close ()
  raises "not owner"            (** caller cannot close *)
end
\end{lstlisting}
Listing~\ref{lst:contract-module-example} shows a contract module for
the auction contract.  It declares the entrypoint \lstinline/bid/ as
\lstinline/paid/, which means that it needs to be invoked with a
non-zero amount of tokens, it gives the pattern \lstinline/()/ for the
input value of type \lstinline/unit/, and it specifies two possible
failure messages that we wish to deal with programmatically.
The \lstinline/close/ entrypoint is similar, but requires no tokens.

\begin{lstlisting}[float,captionpos=b,caption={Generated signature},label={lst:generated-signature}]
module type MyContract = sig
  type bid_errors = 
     | closed       (** auction closed*)
     | too_low      (** bid received is too low *)

  val bid
    : Tezos.pukh -> ?fee:Tezos.tz -> amount:Tezos.tz
    -> (Tezos.status, bid_errors) Tezos.monad

  type close_errors = 
     | not_owner    (** caller cannot close the auction *)

  val close
    : Tezos.pukh -> ?fee:Tezos.tz -> ?amount:Tezos.tz
    -> (Tezos.status, close_errors) Tezos.monad
end
\end{lstlisting}
Listing~\ref{lst:generated-signature} contains an OCaml module
signature as it could be generated from the contract module. The
module \lstinline/Tezos/ supposedly contains types and other low-level
Tezos-specific definitions. Specifically, the type \lstinline/pukh/ for public
key hashes identifies contracts, the type \lstinline/tz/ stands for
Tezos tokens, the type \lstinline/status/ reflects the internal return
status, and \lstinline/monad/ is an internal monad type. The signature
declares a function and an error type for each entrypoint.

The error types mirror the raises clauses. The first argument of each
function is the address of the contract, then an optional argument for
the transaction fee, an argument for passing an amount of tokens
(optional for an unpaid entrypoint), the next argument would be for
the parameter; it is omitted here because its type is
\lstinline/unit/. The return type refers to the specific error type.

\section{Simple Checking}
\label{sec:check-contr-against}

We plan to check the contract by symbolic execution against its
specification in the contract module. Here are some examples of
checkable properties.

For each entrypoint, we collect the set of reachable instructions.
For example, the  \lstinline/AMOUNT/ instruction obtains the amount of
tokens sent with a contract invocation.
It should not be possible to reach that instruction from an unpaid
entrypoint  like \lstinline/close/.

For the \lstinline/FAILWITH/ instructions, we also collect their
arguments. The symbolic interpreter needs to retain concrete values as
much as possible to obtain precise results at this point. Each
argument to \lstinline/FAILWITH/ should be accounted for by one
\lstinline/raises/ clause.

\section{Advanced Checking}
\label{sec:advanced-checking}
\begin{lstlisting}[float,captionpos=b,caption={Safer contract module},label={lst:safer-contract-module}]
contract type SaferAuction = sig
  storage (Pair (bidding : bool) 
                (Pair (owner : address) (hi_bidder : address)))

  entrypoint close ()
  requires (sender = owner) raises "not owner"

  paid entrypoint bid () 
  requires bidding raises "closed" (** auction closed*)
  requires (amount > balance)
  raises "too low" as too_low      (** bid too low *)
end
\end{lstlisting}

As it is expensive to invoke a contract just to find out that it
fails, we propose to extend entrypoint specifications with
preconditions as shown in Listing~\ref{lst:safer-contract-module}. The
idea is that the generated OCaml module tries to check the
preconditions off-chain before invoking the contract. To this end, the
code needs to obtain properties like balance, storage etc of the
contract, but this information is available from the Tezos node without a fee!
We discuss two of the preconditions to highlight the properties that 
need to be analyzed. 

The precondition \lstinline/sender = owner/ of \lstinline/close/ can
be checked off-chain because the owner's address is part of the
storage. However, it is in general unsound to perform such a test
off-chain because the owner's address could change if an entrypoint
changes that component of the storage. To safely check this
precondition, the analysis must determine that the owner component of
the storage remains the same across all possible execution paths of
the contract.

Moreover, the gathering of instructions must build path predicates,
such that each \lstinline/FAILWITH/ instruction comes with a predicate
that must be true to reach the instruction. In contract
implementation, the path predicate is \lstinline/sender != owner/. As
the conjunction of path predicate and precondition is unsatisfiable
and because the \lstinline/owner/ component is constant, an off-chain
test for \lstinline/sender = owner/ precisely predicts whether the
failure condition arises.

The situation is slightly more complex at the \lstinline/bid/
entrypoint. The failure \lstinline/"closed"/ is guarded by
\lstinline/bidding/. As the \lstinline/bidding/ component of the state can change,
precise prediction is not possible. A closer look reveals some
subtlety. If \lstinline/bidding/ is true, then the flag may have
changed by some interleaved call to \lstinline/close/. However, if
\lstinline/bidding/ is false, then there is no point in invoking the
contract because \lstinline/bidding/ will never be reset to true.

Hence the analysis should also record value transitions in the
storage for non-failing executions. For \lstinline/bidding/
\lstinline/bid/ transitions from true to true and \lstinline/close/
transitions from true to false. Thus, if the off-chain check finds
\lstinline/bidding/ $=$ false, then we can precisely predict that
\lstinline/bid/ would fail and trigger the corresponding error without
invoking the contract proper.

For the failure \lstinline/"to low"/, the analysis is very similar:
we need to know that there is no successful execution of
\lstinline/bid/ after an execution of \lstinline/close/. Moreover,
each invocation of \lstinline/bid/ raises the balance of the contract
monotonically. Thus, if the off-chain check
\lstinline/amount > balance/ fails, we can be sure that the contract
invocation will also fail; either because some closed the auction or
because the balance is at least as high as in the off-chain sample. 

%%
%% Bibliography
%%

%% Please use bibtex, 

\bibliography{biblio}

\end{document}
