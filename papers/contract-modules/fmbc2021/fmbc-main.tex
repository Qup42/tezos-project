\documentclass[a4paper,USenglish,american,cleveref, autoref, thm-restate]{oasics-v2021}
%This is a template for producing OASIcs articles. 
%See oasics-v2021-authors-guidelines.pdf for further information.
%for A4 paper format use option "a4paper", for US-letter use option "letterpaper"
%for british hyphenation rules use option "UKenglish", for american hyphenation rules use option "USenglish"
%for section-numbered lemmas etc., use "numberwithinsect"
%for enabling cleveref support, use "cleveref"
%for enabling autoref support, use "autoref"
%for anonymousing the authors (e.g. for double-blind review), add "anonymous"
%for enabling thm-restate support, use "thm-restate"
%for enabling a two-column layout for the author/affilation part (only applicable for > 6 authors), use "authorcolumns"
%for producing a PDF according the PDF/A standard, add "pdfa"

\listfiles

%\graphicspath{{./graphics/}}%helpful if your graphic files are in another directory

\bibliographystyle{plainurl}% the mandatory bibstyle

\title{Towards Contract Modules for the Tezos Blockchain} %TODO Please add

% \titlerunning{Dummy short title} %TODO optional, please use if title is longer than one line

\author{Thi Thu Ha Doan}{University of Freiburg,
  Germany}{doanha@cs.uni-freiburg.de}{https://orcid.org/0000-0002-1825-0097}{supported
by a grant from the Tezos foundation}%TODO mandatory, please use full name; only 1 author per \author macro; first two parameters are mandatory, other parameters can be empty. Please provide at least the name of the affiliation and the country. The full address is optional

\author{Peter Thiemann}{University of Freiburg, Germany}{thiemann@acm.org}{https://orcid.org/0000-0002-9000-1239}{}

\authorrunning{T.T.H Doan and P. Thiemann} %TODO mandatory. First: Use abbreviated first/middle names. Second (only in severe cases): Use first author plus 'et al.'

\Copyright{Thi Thu Ha Doan and Peter Thiemann} %TODO mandatory, please use full first names. OASIcs license is "CC-BY";  http://creativecommons.org/licenses/by/3.0/

%\ccsdesc[100]{\textcolor{red}{Replace ccsdesc macro with valid one}}
%%TODO mandatory: Please choose ACM 2012 classifications from
%%https://dl.acm.org/ccs/ccs_flat.cfm
\ccsdesc[100]{}

\keywords{contract API, modules, static checking} %TODO mandatory; please add comma-separated list of keywords

\category{} %optional, e.g. invited paper

\relatedversion{} %optional, e.g. full version hosted on arXiv, HAL, or other respository/website
%\relatedversiondetails[linktext={opt. text shown instead of the URL}, cite=DBLP:books/mk/GrayR93]{Classification (e.g. Full Version, Extended Version, Previous Version}{URL to related version} %linktext and cite are optional

%\supplement{}%optional, e.g. related research data, source code, ... hosted on a repository like zenodo, figshare, GitHub, ...
%\supplementdetails[linktext={opt. text shown instead of the URL}, cite=DBLP:books/mk/GrayR93, subcategory={Description, Subcategory}, swhid={Software Heritage Identifier}]{General Classification (e.g. Software, Dataset, Model, ...)}{URL to related version} %linktext, cite, and subcategory are optional

%\funding{(Optional) general funding statement \dots}%optional, to capture a funding statement, which applies to all authors. Please enter author specific funding statements as fifth argument of the \author macro.

% \acknowledgements{I want to thank \dots}%optional

%\nolinenumbers %uncomment to disable line numbering

%\hideOASIcs %uncomment to remove references to OASIcs series (logo, DOI, ...), e.g. when preparing a pre-final version to be uploaded to arXiv or another public repository

%Editor-only macros:: begin (do not touch as author)%%%%%%%%%%%%%%%%%%%%%%%%%%%%%%%%%%
\EventEditors{John Q. Open and Joan R. Access}
\EventNoEds{2}
\EventLongTitle{42nd Conference on Very Important Topics (CVIT 2016)}
\EventShortTitle{CVIT 2016}
\EventAcronym{CVIT}
\EventYear{2016}
\EventDate{December 24--27, 2016}
\EventLocation{Little Whinging, United Kingdom}
\EventLogo{}
\SeriesVolume{42}
\ArticleNo{23}
%%%%%%%%%%%%%%%%%%%%%%%%%%%%%%%%%%%%%%%%%%%%%%%%%%%%%%

%% structure
\newcommand{\Angle}[1]{\langle#1\rangle}

%% values
\newcommand\SUNIT{\textbf{()}}
\newcommand{\TRUE}{\textbf{True}}
\newcommand{\FALSE}{\textbf{False}}

%% names
\newcommand{\ALS}{\textbf{als}}
\newcommand{\PAK}{\textbf{pak}}
\newcommand{\PUK}{\textbf{puk}}
\newcommand{\PKH}{\textbf{pkh}}
\newcommand{\PUH}{\textbf{puh}}
\newcommand{\CODE}{\textbf{code}}
\newcommand{\BAL}{\textbf{bal}}
\newcommand{\COU}{\textbf{cou}}
\newcommand{\STORAGE}{\textbf{storage}}
\newcommand{\OP}{\textbf{op}}
\newcommand{\OPH}{\textbf{oph}}
\newcommand{\TIME}{\textbf{t}}
\newcommand{\CONTRACTORS}{\textbf{T}}
\newcommand{\PENDING}{\textbf{P}}
\newcommand{\ACCEPTED}{\textbf{A}}
\newcommand{\MANAGERS}{\textbf{K}}
%% operations
\newcommand{\TRANSFER}[5][\SUNIT]{\text{transfer $#2$ from $#3$ to $#4$ arg $#1$ fee $#5$}}
\newcommand{\ORIGINATE}[6]{\text{originate contract $#1$ transferring $#2$ from $#3$ running $#4$ init $#5$ fee $#6$}}
\newcommand{\NTEZ}{\textbf{n}}
\newcommand{\MTEZ}{\textbf{m}}
\newcommand{\ID}{\textbf{id}}
\newcommand\STRING{\textbf{s}}
%% queries
\newcommand{\QRY}{\textbf{qry}}
\newcommand{\GETBALANCE}[1]{\text{get balance for $#1$}}
\newcommand{\GETSTATUS}[1]{\text{get status for $#1$}}
\newcommand{\GETSTORAGE}[1]{\text{get contract storage $#1$}}
\newcommand{\GETCODE}[1]{\text{get code for $#1$}}
\newcommand{\GETPUBLICKEY}[1]{\text{get public key for $#1$}}
\newcommand{\GETCOUNTER}[1]{\text{get counter for $#1$}}

\newcommand{\ACCOUNTS}{\textbf{C}}
\newcommand{\OPERATIONS}{\textbf{O}}
\newcommand{\CONTRACTS}{\textbf{S}}

\newcommand{\NODE}{\textbf{N}}
\newcommand{\BLOCKCHAIN}{\textbf{B}}

%% functions
\newcommand{\CHECKACC}{\textup{checkAcc}}
\newcommand{\CHECKID}{\textup{checkId}}
\newcommand{\CHECKBAL}{\textup{checkBal}}
\newcommand{\CHECKCOU}{\textup{checkCou}}
\newcommand{\CHECKPUB}{\textup{checkPub}}
\newcommand{\UPDATECOU}{\textup{updateCou}}

\newcommand{\GENERATEOPH}{\textup{generateOph}}

%% transition relations
\newcommand{\NodeTrans}{\longrightarrow_N}
\newcommand{\SystemTrans}{\longrightarrow}


\begin{document}

\maketitle

%TODO mandatory: add short abstract of the document
\begin{abstract}
  Programmatic interaction with a blockchain is often clumsy.
  Many interfaces handle only loosely structured data, often in JSON
  format, that is inconvenient to handle and offers few guarantees.

  Contract modules provide a statically checked interface to interact
  with contracts on the Tezos blockchain. A module specification
  provides all types as well as information about potential failure
  conditions of the contract. The specification is checked against the
  contract implementation using symbolic execution. An OCaml module is
  generated that contains a function for each entrypoint of the
  contract. The types of these functions fully describe the interface
  including the failure conditions and guarantee type-safe and
  sometimes fail-safe invocation of the contract on the blockchain.
\end{abstract}

\section{Introduction}
\label{sec:introduction}

Contracts on the blockchain rarely run in isolation. To be useful
beyond shuffling tokens between user accounts, they need to interact
with the outside world. On the other hand, the outside world also
needs to interact by initiating transactions and starting contracts
that feed information into the blockchain. One direction is addressed
by oracles that watch certain events on the blockchain, create a
response by calculation or gathering data, and then invoke a callback
contract to inject this response into the chain. Trust is an essential
aspect for an oracle.

The other direction is about automatizing certain processes in
connection with the blockchain. For example, opening or closing an
auction according to a schedule, programming a strategy for an
auction, or creating an NFT. To this end, an interface is needed to
invoke contracts safely. Existing interfaces are lacking because they
are essentially untyped (string-based or JSON-based) and often low
level because they require dealing directly with RPC interfaces. Trust
is not needed because the process runs on behalf of a certain user.

Contract modules provide a clean, language-integrated way to interact
with a blockchain. They abstract over underlying string-based
interfaces and details like fee handling. They provide a high-level
typed interface which reduces a contract invocation to a function call
in the language.

Contract modules do not provide a fixed API, but
rather generate a specialized interface for each contract. This
interface is statically checked against the contract implementation to
ensure type safety and exception safety (every failure condition
arising is handled by proper error reporting).

Our work is situated in the context of the Tezos blockchain, which
supports Michelson as its low-level contract language, and the
language OCaml, which supports a powerful module concept that we
enhance with contract modules. 

\section{Context}
\label{sec:context}

Tezos is a third generation, account-based, self amendable
blockchain \cite{tezos-whitepaper}. It employs a proof-of-stake consensus protocol, which
includes ways to evolve the protocol itself. The consensus protocol is
executed by so-called bakers and their proposed blocks are checked by
validators. They receive some compensation in the form of
tokens (Tezzies) for their work.

Each Tezos contract owns an account as well as some storage. Contracts
are pure functions of type parameter $\times$ storage $\to$ operation
list $\times$ storage. When a contract is invoked with a parameter,
the blockchain provides the current storage and updates it with the
second, storage component of its return value. The first component is
a list of blockchain operations (contract deployments, token
transfers, contract invocations) that are executed transactionally
after the first invocation terminates. Each invocation may be
accompanied with an amount of tokens that are added to the current
account balance of the contract.

Contracts are implemented in the Michelson language, a fully typed
stack-based language. Each contract has fixed types for its parameter
and for its storage. The storage is initialized when the contract is
deployed. Besides primitive types like unit, int, bool, address, and string,
there are pairs, sums, functions, lists, and maps (and many more) that
can serve as types for storage and parameters.

\section{An auction contract}
\label{sec:an-auction-contract}

As a concrete example, we consider an auction contract. 
\begin{lstlisting}[language=michelson,numbers=none,emph={close,bid},emphstyle=\underbar]
parameter (or (unit %close) (unit %bid));
storage (pair bool          # bidding allowed
         (pair address      #  contract owner
          address           # highest bidder's address
        ));
\end{lstlisting}
This contract has two entrypoints, \lstinline/close/ and \lstinline/bid/,
expressed by giving the single parameter a sum type. To call the
entrypoint \lstinline/close/ we use the parameter \lstinline/Left ()/
otherwise we use \lstinline/Right ()/, where \lstinline/()/ is the sole
value of type \lstinline/unit/. Its storage is a nested pair which
contains a boolean flag and two addresses. 

%%
%% Bibliography
%%

%% Please use bibtex, 

\bibliography{oasics-v2021-sample-article}

\end{document}
