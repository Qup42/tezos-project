\clearpage
\appendix

\section{Type Soundness}
\label{sec:type-soundness}

\begin{lemma}[Canonical forms]\label{lemma:canonical-forms}
  Given a set of local accounts $\ACCOUNTS$, a blockchain
  $\BLOCKCHAIN$, and a typed value $\JTypeExpr\cdot \VAL \TYPE$.
  \begin{itemize}
  \item If $\TYPE=\TPUH$, then $\VAL=\PUH$ and $\PUH \in \DOM
    (\BLOCKCHAIN.\CONTRACTORS)$.
  \item If $\TYPE = \TPUK$, then $\VAL=\PUK$ and $\exists\PAK$ such
    that $(\PAK, \PUK) \in \ACCOUNTS$.
  \item If $\TYPE = \TADDR$, then $\VAL$ is $\PUH$ or $\PUK$.
  \item If $\TYPE = \TCONTRACT\ \TYPE_p\ \TYPE_s$, then $\VAL=\PUH$
    and $\PUH \in \DOM (\BLOCKCHAIN.\CONTRACTORS)$ and
    $\BLOCKCHAIN.\CONTRACTORS (\PUH) = (\CODE, \TIME, \BAL, \STORAGE)$
    such that $      \JTypeCode \CODE{ \TPAIR\ \TYPE_p\ \TYPE_s}$ and
    $\JTypeValue \STORAGE { \TYPE_s}$.
  \item If $\TYPE = \TCODE\ \TYPE\ \TYPEU$, then $\VAL = \CODE$ and
    $\JTypeCode \CODE{\TPAIR\ \TYPE\ \TYPEU}$.
  \item If $\TYPE = \TOPH\ \TYPE\ \TYPEU$, then $\VAL = \OPH$ and
    $\OPH\in \DOM (\BLOCKCHAIN.\PENDING)$ and $\BLOCKCHAIN.\PENDING
    (\OPH) = \Angle{\OP, \TIME, \STATUS}$ where $\TYPE=\TYPEU=\TNO$ if
    $\OP$ is a transfer and $\TYPE=\TYPE_p\ne\TNO$, $\TYPEU=\TYPE_s\ne\TNO$ if $\OP
    = \ORIGINATE\NTEZ\PUK\CODE\STRING\MTEZ$ and $\JTypeCode\CODE{
      \TPAIR\ \TYPE_p\ \TYPE_s}$.
  \item If $\TYPE = \TSTATUS$, then $\VAL \in \{\STATUSPENDING , \STATUSINCLUDING (\INT),
    \STATUSTIMEOUT\}$.
  \item If $\TYPE = \TEXCEPTION$, then $\VAL \in \{\ERRPRG, \ERRBAL, \ERRCOUNT, \ERRFEE,
    \ERRPUK, \ERRPUH, \ERRARG, \ERRINIT \}$.
  \item If $\TYPE = \TTEZ$, then $\VAL = \NTEZ$, a token amount.
  \item If $\TYPE = \TNO$, then $\VAL$ can be any syntactic value.
  \item If $\TYPE = \TINT$, then $\VAL = \INT$.
  \item If $\TYPE = \TUNIT$, then $\VAL = \SUNIT$.
  \item If $\TYPE = \TBOOL$, then $\VAL \in \{ \TRUE, \FALSE\}$.
  \item If $\TYPE = \TSTRING$, then $\VAL = \STRING$, a string.
  \item If $\TYPE = \TYPE\to\TYPEU$, then $\VAL = \lambda
    \VARIABLE. \EXPR$. 
  \item If $\TYPE = \TPAIR\ \TYPE\ \TYPEU$, then $\VAL = (\VAL',
    \VAL'')$ where $\JTypeExpr\cdot {\VAL'}\TYPE$ and $\JTypeExpr\cdot
    {\VAL''}\TYPEU$ in context $\ACCOUNTS$ and $\BLOCKCHAIN$.
  \item If $\TYPE = \TLIST\ \TYPE$, then either $\VAL = \NIL$ or $\VAL
    = \CONS\ {\VAL'}\ {\VAL''}$ where $\JTypeExpr\cdot{\VAL'}\TYPE$
    and $\JTypeExpr\cdot{\VAL''}{\TLIST\ \TYPE}$ in context
    $\ACCOUNTS$ and $\BLOCKCHAIN$.
  \item If $\TYPE = \TSUM\ \TYPE\ \TYPEU$, then either $\VAL = \LEFT\
    \VAL'$ where $\JTypeExpr\cdot {\VAL'}\TYPE$ or $\VAL= \RIGHT\
    \VAL''$ where $\JTypeExpr\cdot {\VAL''}\TYPEU$ in context
    $\ACCOUNTS$ and $\BLOCKCHAIN$.
  \item If $\TYPE = \TOPTION\ \TYPE$, then $\VAL = \NONE$ or $\VAL =
    \SOME\ \VAL'$ where $\JTypeExpr\cdot {\VAL'}\TYPE$ in context
    $\ACCOUNTS$ and $\BLOCKCHAIN$.
  \end{itemize}
\end{lemma}

\begin{lemma}[Subterm replacement]\label{lemma:subterm-replacement}
  If $\JTypeExpr\cdot{ \EC\EXPR}\TYPE$,
  $\JTypeExpr\cdot\EXPR \TYPE'$, and
  $\JTypeExpr\cdot{\EXPR'} \TYPE'$, then
  $\JTypeExpr\cdot{ \EC{\EXPR'}}\TYPE$.
\end{lemma}
\begin{proof}
  Induction on evaluation context $\ECN$ making use of the fact that
  an evaluation context does not bind variables.
\end{proof}

\begin{lemma}[Preservation for expressions]\label{lemma:type-preservation-expressions}
  If $\JTypeExpr\cdot{ \EXPR}\TYPE$ and $\EXPR \ExprTrans \EXPR'$,
  then  $\JTypeExpr\cdot{ \EXPR'}\TYPE$.
\end{lemma}
\begin{proof}
  Standard result: type preservation for simply typed lambda calculus
  with pairs, sums, and exceptions. Uses
  Lemma~\ref{lemma:subterm-replacement} for reductions in evaluation
  context. See, for instance, Types in
  Programming Languages by Benjamin Pierce.
\end{proof}

\begin{lemma}[Preservation]
  If $\BLOCKCHAIN[\overline\NODE] \SystemTrans{} \BLOCKCHAIN'[\overline{\NODE'}]$ and
  $\JTypeConfig\Delta {\BLOCKCHAIN[\overline\NODE]}$, then there is some
  $\Delta' \supseteq \Delta$ such that
  $\JTypeConfig{\Delta'}{ \BLOCKCHAIN'[\overline{\NODE'}]}$.
\end{lemma}
\begin{proof}
  The proof is by induction on the reduction relation
  $\BLOCKCHAIN[\overline\NODE] \SystemTrans{}
  \BLOCKCHAIN'[\overline\NODE']$ and inversion of the typing
  judgments. We only consider the exemplary reductions shown in the
  paper. We mark all components that belong to the reductum with 
  $'$ as in $\NODE'$.

  From $\JTypeConfig\Delta {\BLOCKCHAIN[\overline\NODE]}$ we obtain
  \begin{gather}
    \label{eq:1}
    \JTypeBlockchain\Delta\BLOCKCHAIN
    \\\label{eq:2}
    \JTypeNode{\NODE_i}
  \end{gather}
  From~\eqref{eq:1} we obtain
  \begin{gather}
    \label{eq:3}
    \DOM (\Delta) = \DOM (\BLOCKCHAIN.\CONTRACTORS)
    \intertext{and $\forall\PUH\in\DOM (\Delta)$}\label{eq:4}
    \Delta (\PUH) = \TPAIR\ \TYPE_p\ \TYPE_s
    \\\label{eq:5}
    \JTypeCode{\BLOCKCHAIN.\CONTRACTORS (\PUH).\CODE} {\TPAIR\ \TYPE_p\ \TYPE_s}
    \\
    \JTypeValue{\BLOCKCHAIN.\CONTRACTORS (\PUH).\STORAGE}{ \TYPE_s}
  \end{gather}
  From~\eqref{eq:2} we obtain, if $\NODE_i = [\overline{\EXPR_i},
  \ACCOUNTS_i]$, 
  \begin{gather}
    \label{eq:6}
    \JTypeExpr\cdot{ \EXPR_{ij}}\TUNIT
  \end{gather}

  \textbf{Reduction }$  \inferrule[Config-Node]
  {\NODE_0 \NodeTrans \NODE_0'}
  { {\BLOCKCHAIN[\NODE_0 :: \overline\NODE]}
    \SystemTrans
    {\BLOCKCHAIN[\NODE_0' :: \overline\NODE]}}$.

  The only possible reduction here is $  \inferrule[Node-Eval]
  {
    \EXPR \ExprTrans \EXPR'
  }{
    [\EC\EXPR :: \EXPRS, \ACCOUNTS] \NodeTrans{}
    [\EC{\EXPR'} :: \EXPRS, \ACCOUNTS]
  }$.

  From~\eqref{eq:6}, we know that $\JTypeExpr\cdot{ \EXPR}\TUNIT$. By
  Lemma~\ref{lemma:type-preservation-expressions}, $\JTypeExpr\cdot{
    \EXPR'}\TUNIT$, the types of the $\EXPRS$ are not affected, hence
  $\JTypeNode{\NODE_0' = [\EC{\EXPR'} :: \EXPRS, \ACCOUNTS]}$. None of the other
  nodes changed, neither did $\BLOCKCHAIN$, so that
  $\JTypeConfig\Delta{{\BLOCKCHAIN[\NODE_0' :: \overline\NODE]}}$. 

  \clearpage
  \textbf{Reduction }$\inferrule[Config-System]{
    \NODE\|\BLOCKCHAIN \SystemTrans \NODE'\|\BLOCKCHAIN'
}{
    \BLOCKCHAIN[\NODE :: \overline{\NODE}] \SystemTrans
    \BLOCKCHAIN'[\NODE' :: \overline{\NODE}]
  }$.

  We need to consider the cases for $\SystemTrans$.

  \textbf{Subcase }$  \inferrule[Node-Inject]{
    \Angle{\PAK,\PUK} \in \ACCOUNTS \\
    \CHECKBAL (\MANAGERS, \PUK, \NTEZ, \MTEZ) \\
    \CHECKARG (\CONTRACTORS, \PUH, \PARAMETER) \\
    \CHECKCOU (\MANAGERS, \PUK) \\
    \CHECKPUH (\CONTRACTORS, \PUH) \\
    \CHECKGAS (\CONTRACTORS, \PUH, \PARAMETER, \MTEZ) \\
    \OPH = \GENERATEOPH (\OP, \TIME) \\
    \OP = \TRANSFER[\PARAMETER]\NTEZ\PUK{\PUH}\MTEZ    
  }{
    { [\EC\OP :: \EXPRS, \ACCOUNTS] \|
      [\PENDING, \MANAGERS, \CONTRACTORS, \TIME] } \SystemTrans \\
    { [\EC{\OPH}  :: \EXPRS, \ACCOUNTS] \|
      [ \OPH \mapsto \Angle{\OP, \TIME, \STATUSPENDING}
      ::\PENDING,
      \UPDATECOU(\MANAGERS, \PUK, \TRUE),
      \CONTRACTORS,
      \TIME]
    }
  }$.
  Here $\NODE = [\EC\OP :: \EXPRS, \ACCOUNTS]$. We first check type
  preservation for the expression part. There are two typing rules for
  the transfer
  $\OP$, but only the one for contract invocation applies as the other
  one requires $\JTypeExpr\cdot{\PUH}\TPUK$, which does not hold.

  % For a token transfer
  % \begin{gather}
  %   \label{eq:7}
  %       \inferrule{
  %     \JTypeExpr\TEnv{\NTEZ}\TTEZ \\
  %     \JTypeExpr\TEnv{\PUK}\TPUK \\
  %     \JTypeExpr\TEnv{\PUH}\TPUK \\
  %     % \JTypeExpr\TEnv{\EXPR_3}{\TCONTRACT\ \TUNIT} \\
  %     \JTypeExpr\TEnv{\PARAMETER}\TUNIT \\
  %     \JTypeExpr\TEnv{\MTEZ}\TTEZ }{
  %     \JTypeExpr\TEnv{\TRANSFER[\PARAMETER]{\NTEZ}{\PUK}{\PUH}{\MTEZ}}\TOPH\
  %     \TNO\ \TNO
  %   }
  % \end{gather}

  For a contract invocation (specialized to empty environment)
  \begin{gather}
    \label{eq:8}
        \inferrule{
      \JTypeExpr\cdot{\NTEZ}\TTEZ \\
      \JTypeExpr\cdot{\PUK}\TPUK \\
      \JTypeExpr\cdot{\PUH}\TCONTRACT\ \TYPE_p\ \TYPE_s \\
      \JTypeExpr\cdot{\PARAMETER}{\TYPE_p} \\
      \JTypeExpr\cdot{\MTEZ}\TTEZ }{
      \JTypeExpr\cdot{\TRANSFER[\PARAMETER]{\NTEZ}{\PUK}{\PUH}{\MTEZ}}{\TOPH\
      \TNO\ \TNO}
    }
  \end{gather}
  The canoncical forms lemma~\ref{lemma:canonical-forms} is parameterized over the accounts
  $\ACCOUNTS$ of the local node and the current contractors
  $\BLOCKCHAIN.\CONTRACTORS$. Hence, we know that the arguments are
  legal, which is also checked by the rule.

  The reduct returns an operation hash $\OPH$ at type $\TOPH\ \TNO\
  \TNO$, which places no restrictions on the context of $\OPH$.

  Moreover, $\Delta' = \Delta$ and $\CONTRACTORS' = \CONTRACTORS$ as
  no new contract is originated.

  We conclude with Lemma~\ref{lemma:subterm-replacement} and
  reapplying \TirName{Config-System}.

  \textbf{Subcase }$\inferrule[Contract-Yes]{
    \JTypeCode\CODE{\TPAIR\ \TYPE\ \TYPEU} \\
    \BLOCKCHAIN.\CONTRACTORS (\PUH) =  \Angle{\CODE, \tilde\TIME, \NTEZ', \STRING'}
    % 
  }{
    { [\EC{\CAST\PUH\TPUH{\TCONTRACT\ \TYPE}}  :: \EXPRS, \ACCOUNTS] \|
      \BLOCKCHAIN
    }
    \SystemTrans
    { [\EC{\PUH} :: \EXPRS, \ACCOUNTS] \|
      \BLOCKCHAIN
    }
  }$.

  Immediate using Lemma~\ref{lemma:canonical-forms} and
  Lemma~\ref{lemma:subterm-replacement}.

  \textbf{Subcase }$\inferrule[Contract-No]{
    % \ADDR \in \DOM (\BLOCKCHAIN.\MANAGERS) \Rightarrow \TYPE\ne\TUNIT \\
    \BLOCKCHAIN.\CONTRACTORS (\PUH) = \Angle{\CODE, \tilde\TIME,
      \NTEZ', \STRING'} \Rightarrow {}
    \JTypeCode\CODE{\TPAIR\ \TYPE'\ \TYPEU} \wedge \TYPE \ne \TYPE' \\
  }{
    { [\EC{\CAST\PUH\TPUH{\TCONTRACT\ \TYPE}}  :: \EXPRS, \ACCOUNTS] \|
      \BLOCKCHAIN
    }
    \SystemTrans
    { [\EC{\RAISE\ \ERRPRG} :: \EXPRS, \ACCOUNTS] \|
      \BLOCKCHAIN
    }
  }$.

  The typing rule for $\RAISE$ can return any type. Hence, this is
  immediate by Lemma~\ref{lemma:subterm-replacement}.

  \textbf{Subcase }$\inferrule[Block-Originate]{
    \Angle{\PAK,\PUK} \in \ACCOUNTS \\ \CHECKBAL (\MANAGERS, \PUK, \NTEZ, \MTEZ) \\
    \CHECKCOU (\MANAGERS, \PUK) \\
    \CHECKPRG (\CODE) \\
    \CHECKGAS (\CODE, \INIT, \NTEZ, \MTEZ)  \\
    \CHECKINIT (\CODE, \STRING) \\
    % \OPH = \GENERATEOPH(\PUK, \CODE, \STRING, \NTEZ, \MTEZ, \TIME)
    % \\
    \OPH = \GENERATEOPH(\OP, \TIME) \\
    \OP = \ORIGINATE\NTEZ\PUK\CODE\STRING\MTEZ }{ [\EC\OP :: \EXPRS,
    \ACCOUNTS
    ] \| [\PENDING, \MANAGERS, \CONTRACTORS, \TIME] \SystemTrans \\
    [\EC{\OPH} :: \EXPRS, \ACCOUNTS] \| [\OPH \mapsto \Angle{\OP,
      \TIME, \STATUSPENDING} ::\PENDING,
    \UPDATECOU(\MANAGERS,\PUK,\TRUE), \CONTRACTORS, \TIME] }$. 
  
  Suppose that $\JTypeCode\CODE {\TPAIR\ \TYPE_p\ \TYPE_s}$. Then
  $\JTypeExpr\cdot\OP{\TOPH\ \TYPE_p\ \TYPE_s}$. But this is the type
  of the $\OPH$ in the reductum as it points to $\OP$ in
  $\PENDING$. Hence, the result is immediate by
  Lemma~\ref{lemma:subterm-replacement}.

  % \textbf{Subcase }$    \inferrule[Block-Accept-Query]{
  %   \OP = \ORIGINATE\NTEZ\PUK\CODE\STRING\MTEZ \\
  %   \PENDING (\OPH) =  \Angle{\OP,
  %     \hat\TIME, \STATUSINCLUDING\ \tilde\TIME} \\
  %   \PUH = \GENERATEHASH (\CODE, \tilde\TIME)
  %   % \CONTRACTORS (\PUH) =  \Angle{\CODE, \tilde\TIME, \NTEZ', \STRING'}
  % }{
  %   [\EC{\GETCONTRACT\ \OPH}  :: \EXPRS, \ACCOUNTS] \| [\PENDING, \MANAGERS, \CONTRACTORS, \TIME]
  %   \SystemTrans 
  %   [\EC{ \PUH}  :: \EXPRS, \ACCOUNTS] \|  [\PENDING, \MANAGERS, \CONTRACTORS, \TIME]
  % }$.

  

  \textbf{Subcase }$\inferrule[Query-Balance-Implicit]{
    % \BLOCKCHAIN =
    % [\PENDING, \PUK \mapsto \Angle{\BAL, \COU} :: \MANAGERS,
    % \CONTRACTORS, \TIME]
    \BLOCKCHAIN.\MANAGERS (\PUK) = \Angle{\BAL,\COU}
  }{[\EC{\GETBALANCE\PUK} :: \EXPRS, \ACCOUNTS] \| \BLOCKCHAIN
    \SystemTrans\ [\EC{\BAL} ::\EXPRS, \ACCOUNTS] \| \BLOCKCHAIN} $.
  
  The reduction replaces $\GETBALANCE\PUK$ of type $\TTEZ$ by $\BAL$
  of the same type. Hence, the result is immediate by
  Lemma~\ref{lemma:subterm-replacement}.

  \textbf{Subcase }$\inferrule[Query-Balance-Fail]{ \PUK \notin \DOM
    (\BLOCKCHAIN.\MANAGERS) \ }{[\EC{\GETBALANCE\PUK} :: \EXPRS,
    \ACCOUNTS] \| \BLOCKCHAIN \SystemTrans {[\EC{\RAISE\ \ERRPUK}
      ::\EXPRS, \ACCOUNTS] \| \BLOCKCHAIN}}$.

  Immediate by Lemma~\ref{lemma:subterm-replacement} because $\RAISE$
  can have any type.
    
  \clearpage
  \textbf{Reduction }$\inferrule[Config-Block]
  {\BLOCKCHAIN \BlockTrans \BLOCKCHAIN'}
  { \BLOCKCHAIN[{\overline\NODE}]
    \SystemTrans
    \BLOCKCHAIN'[{\overline\NODE}]}$.

  We need to considere cases for $\BlockTrans$.

  \textbf{Subcase }$\inferrule[Block-Accept]{
    \OP = \TRANSFER[\PARAMETER]\NTEZ\PUK{\PUH}\MTEZ \\
    \TIME - \hat\TIME \le 60
  }{
    { 
      [\OPH \mapsto \Angle{\OP, \hat \TIME, \STATUSPENDING}
      ::\PENDING, \MANAGERS,
      \CONTRACTORS, \TIME]}
    \BlockTrans 
    {
      [\OPH \mapsto \Angle{\OP, \hat\TIME, \STATUSINCLUDING\ \TIME} :: \PENDING}, \\
    { \UPDATESUCC (\MANAGERS, \PUK, \NTEZ, \MTEZ), 
      \UPDATECONSTR (\CONTRACTORS, \PUH, \NTEZ, \PARAMETER), \TIME +1]
    }
  }$.

  No typing-related properties are affected.

  \textbf{Subcase }$\inferrule[Block-Originate-Accept]{
    \OP = \ORIGINATE\NTEZ\PUK\CODE\STRING\MTEZ \\
    \PUH = \GENERATEHASH(\CODE, \TIME) \\
    \TIME-\hat\TIME  \le 60
  }{
    [\OPH \mapsto \Angle{\OP, \hat\TIME, \STATUSPENDING} :: \PENDING, \MANAGERS, \CONTRACTORS, \TIME]
    \BlockTrans \\
    [\OPH \mapsto \Angle{\OP, \hat\TIME, \STATUSINCLUDING\ \TIME} :: \PENDING, \UPDATESUCC
    (\MANAGERS, \PUK, \NTEZ, \MTEZ),\\ \PUH \mapsto  \Angle{\CODE, \TIME, \NTEZ, \STRING} :: \CONTRACTORS, \TIME+1]
  }$.
  
  This reduction extends $\CONTRACTORS$ with a new entry for
  $\PUH$. To preserve typing, we need to extend $\Delta$ with the
  binding $\PUH : \TPAIR\ \TYPE_p\ \TYPE_s$ where $\JTypeCode\CODE
  {\TPAIR\ \TYPE_p\ \TYPE_s}$. The generated code pointer is obtained
  with a query operation via the operation hash $\OPH$, which is also
  connected to the parameter and storage types. 

  \textbf{Subcase }$  \inferrule[Block-Timeout]{
    \TIME-\hat\TIME > 60
  }{ 
    {[\OPH \mapsto \Angle{\OP, \hat \TIME, \STATUSPENDING}
     ::\PENDING, \MANAGERS,
      \CONTRACTORS, \TIME]}
    \BlockTrans \\
    { 
      [\OPH \mapsto \Angle{\OP, \hat \TIME, \STATUSTIMEOUT}
     :: \PENDING,  \UPDATECOU(\MANAGERS, \OP.\PUK, \FALSE),
      \CONTRACTORS, \TIME]}
  }$.

  No typing-related properties are affected.
\end{proof}

\clearpage
\begin{lemma}[Progress for expressions]\label{lemma:progress-expressions}
  If $\JTypeExpr\cdot{ \EXPR}\TYPE$, then either
  \begin{itemize}
  \item $\EXPR$ is a value,
  \item $\EXPR \ExprTrans \EXPR'$, or
  \item $\EXPR = \EC{\EXPR'}$ is a blockchain operation in an evaluation context:
    \begin{itemize}
    \item $\EXPR' = \TRANSFER[\PARAMETER]\NTEZ\PUK{\PUH}\MTEZ$;
    \item $\EXPR' = \ORIGINATE\NTEZ\PUK\CODE\INIT\MTEZ$;
    \item $\EXPR' = \QOP\ \VAL$;
    \item $\EXPR' = \CAST\VAL\TYPE\TYPEU$ where $\TYPEU \SubType \TYPE$.
    \end{itemize}
  \end{itemize}
\end{lemma}
\begin{proof}
  Standard result: progress for simply type lambda calculus with
  pairs, sums, and exceptions. Upcasts are resolved by identity
  reductions. The blockchain operations including downcasts are not
  handled by the $\ExprTrans$ relation. 
\end{proof}
\begin{lemma}[Progress]
  If $\JTypeConfig\Delta{\BLOCKCHAIN[\overline\NODE]}$, then either
  all expressions in all nodes are unit values or there is a
  configuration $\BLOCKCHAIN'[\overline\NODE']$ such that
  $\BLOCKCHAIN[\overline\NODE] \SystemTrans \BLOCKCHAIN'[\overline\NODE']$.
\end{lemma}
\begin{proof}
  From $\JTypeConfig\Delta {\BLOCKCHAIN[\overline\NODE]}$ we obtain
  \begin{gather}
    \label{eq:101}
    \JTypeBlockchain\Delta\BLOCKCHAIN
    \\\label{eq:102}
    \JTypeNode{\NODE_i}
  \end{gather}
  From~\eqref{eq:101} we obtain
  \begin{gather}
    \label{eq:103}
    \DOM (\Delta) = \DOM (\BLOCKCHAIN.\CONTRACTORS)
    \intertext{and $\forall\PUH\in\DOM (\Delta)$}\label{eq:104}
    \Delta (\PUH) = \TPAIR\ \TYPE_p\ \TYPE_s
    \\\label{eq:105}
    \JTypeCode{\BLOCKCHAIN.\CONTRACTORS (\PUH).\CODE} {\TPAIR\ \TYPE_p\ \TYPE_s}
    \\
    \JTypeValue{\BLOCKCHAIN.\CONTRACTORS (\PUH).\STORAGE}{ \TYPE_s}
  \end{gather}
  From~\eqref{eq:102} we obtain, if $\NODE_i = [\overline{\EXPR_i},
  \ACCOUNTS_i]$, 
  \begin{gather}
    \label{eq:106}
    \JTypeExpr\cdot{ \EXPR_{ij}}\TUNIT
  \end{gather}

  For each such $\EXPR_{ij}$, Lemma~\ref{lemma:progress-expressions}
  yields that either
  \begin{itemize}
  \item $\EXPR_{ij}$ is a value; as it has type $\TUNIT$, we obtain
    $\EXPR_{ij} = \SUNIT$ by Lemma~\ref{lemma:canonical-forms};
  \item $\EXPR_{ij} \ExprTrans \EXPR_{ij}'$, in which case the whole
    system makes a step; or
  \item $\EXPR_{ij} = \EC{\EXPR}$ where $\EXPR$ is a blockchain operation.
  \end{itemize}

  \textbf{Subcase } $\EXPR =
  \TRANSFER[\PARAMETER]\NTEZ\PUK{\PUH}\MTEZ$. In this case, the
  \TirName{Node-Inject} reduction is in principle enabled:
  \begin{mathpar}
    \inferrule[Node-Inject]{
      \Angle{\PAK,\PUK} \in \ACCOUNTS \\
      \CHECKBAL (\MANAGERS, \PUK, \NTEZ, \MTEZ) \\
      \CHECKARG (\CONTRACTORS, \PUH, \PARAMETER) \\
      \CHECKCOU (\MANAGERS, \PUK) \\
      \CHECKPUH (\CONTRACTORS, \PUH) \\
      \CHECKGAS (\CONTRACTORS, \PUH, \PARAMETER, \MTEZ) \\
      \OPH = \GENERATEOPH (\OP, \TIME) \\
      \OP = \TRANSFER[\PARAMETER]\NTEZ\PUK{\PUH}\MTEZ    
    }{
      { [\EC\OP :: \EXPRS, \ACCOUNTS] \|
        [\PENDING, \MANAGERS, \CONTRACTORS, \TIME] } \SystemTrans \\
      { [\EC{\OPH}  :: \EXPRS, \ACCOUNTS] \|
        [ \OPH \mapsto \Angle{\OP, \TIME, \STATUSPENDING}
        ::\PENDING,
        \UPDATECOU(\MANAGERS, \PUK, \TRUE),
        \CONTRACTORS,
        \TIME]
      }
    }
  \end{mathpar}
  Thanks to the canonical forms Lemma~\ref{lemma:canonical-forms}, we
  know that $\Angle{\PAK,\PUK} \in \ACCOUNTS$, $\CHECKARG
  (\CONTRACTORS, \PUH, \PARAMETER)$ holds, and $\CHECKPUH
  (\CONTRACTORS, \PUH)$ holds. If one of the remaining checks fails,
  then one of the \TirName{Node-Reject} transitions throws an
  exception, so the configuration steps in every case. 

  \textbf{Subcase } $\EXPR = \ORIGINATE\NTEZ\PUK\CODE\INIT\MTEZ$. In
  this case, the \TirName{Block-Originate} reduction is in principle
  enabled:
  \begin{mathpar}
    \inferrule[Block-Originate]{
      \Angle{\PAK,\PUK} \in \ACCOUNTS \\ \CHECKBAL (\MANAGERS, \PUK, \NTEZ, \MTEZ) \\
      \CHECKCOU (\MANAGERS, \PUK) \\
      \CHECKPRG (\CODE) \\
      \CHECKGAS (\CODE, \INIT, \NTEZ, \MTEZ)  \\
      \CHECKINIT (\CODE, \STRING) \\
      % \OPH = \GENERATEOPH(\PUK, \CODE, \STRING, \NTEZ, \MTEZ, \TIME)
      % \\
      \OPH = \GENERATEOPH(\OP, \TIME) \\
      \OP = \ORIGINATE\NTEZ\PUK\CODE\STRING\MTEZ }{ [\EC\OP :: \EXPRS,
      \ACCOUNTS
      ] \| [\PENDING, \MANAGERS, \CONTRACTORS, \TIME] \SystemTrans \\
      [\EC{\OPH} :: \EXPRS, \ACCOUNTS] \| [\OPH \mapsto \Angle{\OP,
        \TIME, \STATUSPENDING} ::\PENDING,
      \UPDATECOU(\MANAGERS,\PUK,\TRUE), \CONTRACTORS, \TIME] }
  \end{mathpar}
  Thanks to the canonical forms Lemma~\ref{lemma:canonical-forms}, we
  know that $\Angle{\PAK,\PUK} \in \ACCOUNTS$, $\CHECKPRG (\CODE)$ holds, and $\CHECKINIT
  (\CODE, \STRING)$ holds. If one of the remaining checks fails,
  then one of the \TirName{Node-Reject} transitions throws an
  exception, so the configuration steps in every case. 
  
  \textbf{Subcase } $\EXPR = \QOP\ \VAL$. If $\EXPR =
  \GETBALANCE\VAL$, then inversion tells us that $\JTypeExpr\cdot{
    \VAL}\TADDR$ and by canonical forms
  (Lemma~\ref{lemma:canonical-forms}), it must be that $\VAL$ has the
  form $\PUK$ or $\PUH$. In any case, the value is a meaningful
  address for the manager $\MANAGERS$. Depending on whether the
  address is in use, one of the reductions
  \TirName{Query-Balance-Implicit} or \TirName{Query-Balance-Fail} can
  execute. There are further analogous reductions handling the case where
  $\VAL=\PUH$ and we ask for the balance of a smart contract.

  Most queries behave like $\GETBALANCE\cdot$, except getting a
  contract handle from an operation hash:

  \textbf{Subcase }$\EXPR = \GETCONTRACT\VAL$. This query is somewhat
  special as it is handled with reduction
  \TirName{Block-Accept-Query}. By inversion and canonical forms
  (Lemma~\ref{lemma:canonical-forms}) we know that $\VAL = \OPH$ is a
  valid operation hash of type $\TCONTRACT\ \TYPE\ \TYPEU$ where
  $\TYPE\ne\TNO$ and $\TYPEU \ne\TNO$.

  However, this reduction is conditional on the state of the
  transaction; it requires the new contract to have status
  $\STATUSINCLUDED$. If the contract has status $\STATUSTIMEOUT$, then
  the query raises and exception, analogous to the
  \TirName{Query-Balance-Fail} reduction. If the contract has status
  $\STATUSPENDING$, then the expression is blocked, but the system can
  make a step using \TirName{Block-Originate-Accept} that changes the
  status from $\STATUSPENDING$ to $\STATUSINCLUDED$. Alternatively,
  \TirName{Block-Timeout} can make a step to change the status to
  $\STATUSTIMEOUT$. In any case, the system as a whole can make a reduction.

  \textbf{Subcase } $\EXPR = \CAST\VAL\TYPE\TYPEU$ where $\TYPEU \SubType \TYPE$.
  As an example, we consider the reductions \TirName{Contract-Yes} and
  \TirName{Contract-No}, where a cast is applied to a value of type
  $\TPUH$. By canonical forms, we know that the value has the
  form $\PUH \in \DOM (\BLOCKCHAIN.\CONTRACTORS)$. The code pointed to
  by this hash is checked at run time and results either in a $\PUH$
  at suitable contract type (-Yes reduction) or in raising an
  exception (-No reduction). 

\end{proof}


%%% Local Variables:
%%% mode: latex
%%% TeX-master: "paper"
%%% End:
