\documentclass{article}

\usepackage[utf8]{inputenc}
\usepackage{amsmath}
\usepackage{mathpartir}
\usepackage{listings}

% Copyright 2017 Sergei Tikhomirov, MIT License
% https://github.com/s-tikhomirov/solidity-latex-highlighting/

\usepackage{listings, xcolor}

\definecolor{verylightgray}{rgb}{.97,.97,.97}

\lstdefinelanguage{Solidity}{
	keywords=[1]{anonymous, assembly, assert, balance, break, call, callcode, case, catch, class, constant, continue, constructor, contract, debugger, default, delegatecall, delete, do, else, emit, event, experimental, export, external, false, finally, for, function, gas, if, implements, import, in, indexed, instanceof, interface, internal, is, length, library, log0, log1, log2, log3, log4, memory, modifier, new, payable, pragma, private, protected, public, pure, push, require, return, returns, revert, selfdestruct, send, solidity, storage, struct, suicide, super, switch, then, this, throw, transfer, true, try, typeof, using, value, view, while, with, addmod, ecrecover, keccak256, mulmod, ripemd160, sha256, sha3}, % generic keywords including crypto operations
	keywordstyle=[1]\color{blue}\bfseries,
	keywords=[2]{address, bool, byte, bytes, bytes1, bytes2, bytes3, bytes4, bytes5, bytes6, bytes7, bytes8, bytes9, bytes10, bytes11, bytes12, bytes13, bytes14, bytes15, bytes16, bytes17, bytes18, bytes19, bytes20, bytes21, bytes22, bytes23, bytes24, bytes25, bytes26, bytes27, bytes28, bytes29, bytes30, bytes31, bytes32, enum, int, int8, int16, int24, int32, int40, int48, int56, int64, int72, int80, int88, int96, int104, int112, int120, int128, int136, int144, int152, int160, int168, int176, int184, int192, int200, int208, int216, int224, int232, int240, int248, int256, mapping, string, uint, uint8, uint16, uint24, uint32, uint40, uint48, uint56, uint64, uint72, uint80, uint88, uint96, uint104, uint112, uint120, uint128, uint136, uint144, uint152, uint160, uint168, uint176, uint184, uint192, uint200, uint208, uint216, uint224, uint232, uint240, uint248, uint256, var, void, ether, finney, szabo, wei, days, hours, minutes, seconds, weeks, years},	% types; money and time units
	keywordstyle=[2]\color{teal}\bfseries,
	keywords=[3]{block, blockhash, coinbase, difficulty, gaslimit, number, timestamp, msg, data, gas, sender, sig, value, now, tx, gasprice, origin},	% environment variables
	keywordstyle=[3]\color{violet}\bfseries,
	identifierstyle=\color{black},
	sensitive=false,
	comment=[l]{//},
	morecomment=[s]{/*}{*/},
	commentstyle=\color{gray}\ttfamily,
	stringstyle=\color{red}\ttfamily,
	morestring=[b]',
	morestring=[b]"
}

\lstset{
	language=Solidity,
	backgroundcolor=\color{verylightgray},
	extendedchars=true,
	basicstyle=\footnotesize\ttfamily,
	showstringspaces=false,
	showspaces=false,
	numbers=left,
	numberstyle=\footnotesize,
	numbersep=9pt,
	tabsize=2,
	breaklines=true,
	showtabs=false,
	captionpos=b
}


\title{Expos\'{e} Master Thesis\\
  Checking Assertions for Smart Contracts}
\author{Peter Thiemann}
\begin{document}
\maketitle{}

\section{Introduction}
\label{sec:introduction}

Every computation performed by a Smart Contract on the blockchain generates costs. Each
unit of computation and each unit of storage used by an algorithm must be paid for. To
avoid this cost, users might perform some computation offchain and submit the result as a
parameter to a contract. Typically, such a computation asserts certain properties of the
submitted parameter. 

However, this approach raises the issue that on the one hand the contract should take
advantage of the offchain computation and just assume that the submitted parameter has
these properties, but on the other hand, the offchain computation might be wrong and
submit illegal parameters. So, we need a mechanism that checks the validity of the
assumptions before the contract starts executing.

As an example, consider a contract that takes a prime number as a parameter.
\begin{lstlisting}[numbers=none]
contract Example {
  function (int p) public {
    // assume p is prime
    ...
  }
}
\end{lstlisting}
We express the assumption with an explicit assertion.
\begin{displaymath}
  (\forall n) (2 \le n \le \sqrt p) \Rightarrow (p \mathbin{\%} n) \ne 0
\end{displaymath}
We could test this assumption with a loop in the contract, but the test would take
$O(\sqrt p)$ time (assuming constant time for computing the modulus) and thus produce cost
depending on $p$ accordingly. 
However, we could do better by recruiting the validators of the contract for a distributed
effort to find a counterexample. To this end, we consider the negation of the assertion.
\begin{displaymath}
  (\exists n) (2 \le n \le \sqrt p) \wedge (p \mathbin{\%} n) = 0
\end{displaymath}
This assertion can be checked pointwise by having each validator independently choose a
random $n$ fulfilling $2 \le n \le 
\sqrt p$ and checking whether $(p \mathbin{\%} n) = 0$. If the remainder is $0$, the
validator found a counterexample, posts its veto to the P2P net, and stops further
exection. Otherwise, it accepts $p$ knowing that other points will be checked by other
validators.

In this scenario, each validator only needs to be paid to generate a random number and
perform a division, which is a constant cost independant from $p$.

Of course, this validation is only probabilistic, so its effectiveness depends on the
number of validators. One could say that the community of validators implements a Bloom
filter for the set of primes: if a value $p$ is rejected it is certainly not a prime (because
there exists a counterexample); if a value $p$ is not rejected it is prime with a
probability that depends on $p$ and the number of validators. 

As another example, consider a contract that takes a sorted array of integers.
\begin{lstlisting}[numbers=none]
contract Sorted {
  function find (int[50] a, int v) public {
    // assume a is sorted
  }
}
\end{lstlisting}
The explicit assertion would be
\begin{displaymath}
  (\forall k) (0\le k <49) \Rightarrow a[k] \le a[k+1]
\end{displaymath}
While we can check this contract in $O(1)$ time, the constant factor is large! So we
consider its negation.
\begin{displaymath}
  (\exists k) (0\le k <49) \wedge a[k] > a[k+1]
\end{displaymath}
Again, we can have the validators generate a random number $k$ and check the
condition. If it is true, then the validator found a counterexample for the sortedness
of the array. Otherwise, the validator relies on the others.

Let's assume the array is unsorted only at position $0$, the size of the array is $n$,
and the number of validators is $m$. Each validator independently has a probability of
$1/n$ to detect the problem. The probability that no validator checks at
position $0$ converges to $0$ as the number of validators approaches infinity.
\begin{displaymath}
  \lim_{m\to\infty}\frac{(n-1)^m}{n^m}
  = \lim_{m\to\infty} \left(\frac{n-1}{n}\right)^m
  = 0
\end{displaymath}

\section{Further Thoughts}
\label{sec:further-thoughts}

There may be an alternative way of implementing this scheme without modifying the
protocol, but instead adding instructions to the virtual machine. One instruction would
generate a random number (EVM uses 512 bit unsigned integer IIRC), the other would
raise an overriding failure. Ideally, once a counterexample has been found, this
counterexample is broadcast to the network where it can be verified by further nodes.

This approach poses the danger of a DOS attack. Someone could post a false
counterexample, which does not really lead to a failure. If everyone switches to check
this proposed counterexample instead of generating a random number, then the real
problem may remain hidden. So, there has to be a component in the protocol that
penalizes someone who broadcasts a false counterexample. Moreover, not all validators
should switch to recheck a proposed counterexample; or they could check it in addition
to their own random check.

The interesting assertion formulas are those with prenex universal
quantification. Negation turns them into prenex existential quantification. But it
would be ineffective to generate a random integer and then check that it falls in the
interesting range (e.g., $2\le n \le \sqrt p$ and $0\le k< n-1$). Rather, the formula
needs to be analzed and the random generator should be restricted to only generate
interesting values. 

It would not be significantly more expensive to have every validator perform a
predetermined finite number of checks for each assertion.

What is the minimum number $m$ of validators such that the probability of spotting an
error is less than a given threshold $q$?

Instead of using random numbers, is there a way to coordinate a systematic exploration
of the iteration space? (Is this at all necessary?)

One could also check properties of the values stored by the contract.

Connections to random testing?

\end{document}
