\documentclass{article}

\usepackage[utf8]{inputenc}
\usepackage{amsmath}
\usepackage{mathpartir}
\usepackage{listings}
\usepackage{url}

\usepackage{hyperref}

\usepackage{xcolor}
\usepackage{tcolorbox}
\tcbuselibrary{breakable}

\newenvironment{changethis}{%
  \begin{tcolorbox}[breakable,notitle,boxrule=0pt,colback=blue!20,colframe=blue!20]}{%
  \end{tcolorbox}}

\newenvironment{ethereum}{%
  \begin{tcolorbox}[breakable,notitle,boxrule=0pt,colback=red!20,colframe=red!20]}{%
  \end{tcolorbox}}


% Copyright 2017 Sergei Tikhomirov, MIT License
% https://github.com/s-tikhomirov/solidity-latex-highlighting/

\usepackage{listings, xcolor}

\definecolor{verylightgray}{rgb}{.97,.97,.97}

\lstdefinelanguage{Solidity}{
	keywords=[1]{anonymous, assembly, assert, balance, break, call, callcode, case, catch, class, constant, continue, constructor, contract, debugger, default, delegatecall, delete, do, else, emit, event, experimental, export, external, false, finally, for, function, gas, if, implements, import, in, indexed, instanceof, interface, internal, is, length, library, log0, log1, log2, log3, log4, memory, modifier, new, payable, pragma, private, protected, public, pure, push, require, return, returns, revert, selfdestruct, send, solidity, storage, struct, suicide, super, switch, then, this, throw, transfer, true, try, typeof, using, value, view, while, with, addmod, ecrecover, keccak256, mulmod, ripemd160, sha256, sha3}, % generic keywords including crypto operations
	keywordstyle=[1]\color{blue}\bfseries,
	keywords=[2]{address, bool, byte, bytes, bytes1, bytes2, bytes3, bytes4, bytes5, bytes6, bytes7, bytes8, bytes9, bytes10, bytes11, bytes12, bytes13, bytes14, bytes15, bytes16, bytes17, bytes18, bytes19, bytes20, bytes21, bytes22, bytes23, bytes24, bytes25, bytes26, bytes27, bytes28, bytes29, bytes30, bytes31, bytes32, enum, int, int8, int16, int24, int32, int40, int48, int56, int64, int72, int80, int88, int96, int104, int112, int120, int128, int136, int144, int152, int160, int168, int176, int184, int192, int200, int208, int216, int224, int232, int240, int248, int256, mapping, string, uint, uint8, uint16, uint24, uint32, uint40, uint48, uint56, uint64, uint72, uint80, uint88, uint96, uint104, uint112, uint120, uint128, uint136, uint144, uint152, uint160, uint168, uint176, uint184, uint192, uint200, uint208, uint216, uint224, uint232, uint240, uint248, uint256, var, void, ether, finney, szabo, wei, days, hours, minutes, seconds, weeks, years},	% types; money and time units
	keywordstyle=[2]\color{teal}\bfseries,
	keywords=[3]{block, blockhash, coinbase, difficulty, gaslimit, number, timestamp, msg, data, gas, sender, sig, value, now, tx, gasprice, origin},	% environment variables
	keywordstyle=[3]\color{violet}\bfseries,
	identifierstyle=\color{black},
	sensitive=false,
	comment=[l]{//},
	morecomment=[s]{/*}{*/},
	commentstyle=\color{gray}\ttfamily,
	stringstyle=\color{red}\ttfamily,
	morestring=[b]',
	morestring=[b]"
}

\lstset{
	language=Solidity,
	backgroundcolor=\color{verylightgray},
	extendedchars=true,
	basicstyle=\footnotesize\ttfamily,
	showstringspaces=false,
	showspaces=false,
	numbers=left,
	numberstyle=\footnotesize,
	numbersep=9pt,
	tabsize=2,
	breaklines=true,
	showtabs=false,
	captionpos=b
}


\title{Expos\'{e} Master Thesis\\
  Checking Assertions for Smart Contracts}
\author{Peter Thiemann}
\begin{document}
\maketitle{}

\section{Introduction}
\label{sec:introduction}

Every computation performed by a Smart Contract on the blockchain generates costs. Each
unit of computation and each unit of storage used by an algorithm must be paid for. To
avoid this cost, an application might perform some computation away from the blockchain
(i.e., off-chain) and submit the result as a parameter to a contract on the
blockchain. Typically, such a computation asserts certain properties of the 
submitted parameter. 

However, this approach raises the issue that on the one hand the contract should take
advantage of the offchain computation and assume that the submitted parameter has
these properties, but on the other hand, the off-chain computation might be wrong and
submit illegal parameters. So, we need a mechanism that checks the validity of the
assumptions before the contract starts executing.

As an example, consider a contract that takes a prime number as a parameter.
\begin{lstlisting}[numbers=none]
contract Example {
  function (int p) public {
    // assume p is prime
    ...
  }
}
\end{lstlisting}
This assumption can be expressed with an explicit assertion in predicate logic.
\begin{gather}\label{eq:5}
  (\forall n) (2 \le n \le \sqrt p) \Rightarrow (p \mathbin{\%} n) \ne 0
\end{gather}
To test the validity of this assumption requires a loop in the contract, but the test would take
$O(\sqrt p)$ time (assuming constant time for computing the remainder) and thus produce extra cost
linear in $\sqrt p$ accordingly. 
However, we could do better by recruiting the validators of the contract for a distributed
effort to find a counterexample. To this end, we consider the negation of the assertion.
\begin{gather}\label{eq:4}
  (\exists n) (2 \le n \le \sqrt p) \wedge (p \mathbin{\%} n) = 0
\end{gather}
This assertion can be checked pointwise by having each validator independently choose a
random $n$ fulfilling $2 \le n \le 
\sqrt p$ and checking whether $(p \mathbin{\%} n) = 0$. If the remainder is $0$, the
validator found a counterexample, posts its veto to the P2P net, and stops further
exection. Otherwise, it accepts $p$ knowing that other points will be checked by other
validators.

In this scenario, each validator only needs to be paid to generate a random number and
perform a division, which is a constant cost independent from $p$.

Of course, this validation is only probabilistic, so its effectiveness depends on the
number of validators. One could say that the community of validators implements a Bloom
filter for the set of primes: if a value $p$ is rejected it is certainly not a prime (because
there exists a counterexample); if a value $p$ is not rejected it is prime with a
probability that depends on $p$ and the number of validators. 

As another example, consider a contract that takes a sorted array of integers.
\begin{lstlisting}[numbers=none]
contract Sorted {
  function find (int[50] a, int v) public {
    // assume a is sorted
  }
}
\end{lstlisting}
The explicit assertion would be
\begin{gather}\label{eq:1}
  (\forall k) (0\le k <49) \Rightarrow a[k] \le a[k+1]
\end{gather}
While we can check this contract in $O(1)$ time, the constant factor is big! So we
consider its negation.
\begin{displaymath}
  (\exists k) (0\le k <49) \wedge a[k] > a[k+1]
\end{displaymath}
Again, we can have every validator generate a random number $k$. If the
condition is true for such $k$, then the validator found a counterexample for the sortedness
of the array. Otherwise, the validator relies on the other validators to check
different numbers.

To obtain an estimate of the number of validators needed to find a counterexample with
high probibility, 
let's assume the array is unsorted only at position $0$, the size of the array is $n$,
and the number of validators is $m$. Each validator independently has a probability of
$1/n$ to detect the problem and thus probability $\frac{n-1} n$ not to detect the
problem. Hence, if we assume that each $k$ is chosen independenty from a uniform distribution,
the probability that no validator checks at
position $0$ converges to $0$ as the number of validators approaches infinity.
\begin{displaymath}
  \lim_{m\to\infty}\frac{(n-1)^m}{n^m}
  = \lim_{m\to\infty} \left(\frac{n-1}{n}\right)^m
  = 0
\end{displaymath}

In Dafny (citation) \url{https://rise4fun.com/Dafny/tutorialcontent/guide#h29} you can
write
\begin{lstlisting}
forall j, k :: 0 <= j < k < a.Length ==> a[j] <= a[k]
\end{lstlisting}
to express that an array is sorted. This predicate is equivalent to the one given in
\eqref{eq:1}, but it might be more challenging to test. Its negation is
\begin{gather}
  \label{eq:2}
  (\exists j, k ) (0\le j< k < |a|) \wedge a[j] > a[k]
\end{gather}
So we'd have to generate two random numbers $j$ and $k$ such that the condition $0 \le
j < k < |a|$ is fulfilled.


Another condition that might be tested on an array is the heap condition
\begin{gather}
  \label{eq:3}
  (\forall i) (0 \le i < \lfloor|a|/2\rfloor) \Rightarrow a[i] \le a[2i+1] \wedge (2i+2
  < |a| \Rightarrow a[i] \le a[2i+2])
\end{gather}


\section{Tasks}
\label{sec:tasks}

\subsection{Offline Design}
\label{sec:offline-design}

\begin{itemize}
\item Formally define a set of logical formulas that are amenable to the kind of checking
  described above. In a first approximation, the interesting assertion formulas seem to
  be those with prenex universal quantification.

  In a first iteration, we want to restrict ourselves to formulas of
  predicate logic with prenex universal quantification only.
  \begin{align*}
    t &::= \text{primitive types like int, bool, string, \dots} \\
    \Theta &::= \Phi \mid \forall (x:t) \Theta \\
    \Phi,\Psi &::= \neg\Phi \mid \Phi\wedge\Psi \mid \Phi\vee\Psi \mid
                M \rho N \\
    \rho &::= < \mid > \mid \le \mid \ge \mid = \mid \ne \mid \dots \\
    M, N &::= x \mid c \mid M \oplus N  \mid f (\overline M) \\
    \oplus &::= +\mid -\mid * \mid / \mid \% \mid \dots \\
    c &::= \text{constants: numbers, strings, etc} \\
    f &::= \text{operations (existing in the blockchain VM)}
  \end{align*}
  Later, we might consider existential quantification and user-defined
  functions (for example, sqrt, length of a list, access list element
  are not predefined by the VM and have to be implemented. The list
  operations would be really good to have, though). See
  Section~\ref{sec:suggestions-tezos} for suggestions for a concrete syntax.
\item Collect examples for interesting formulas.
\item Formalize the cost incurred by checking a formula.
\item Extend the contract language with assertions stating these logical
  formulas. There are several approaches to perform this extension.
  \begin{enumerate}
  \item (Solidity) Extend the syntax of the assert statement. Requires an extension of
    the compiler.
  \item (Tezos/Solidity) The assertion is stated in its own syntax in a file separate from the
    contract implementation. Requires a dedicated compiler of assertions to
    Michelson. The generated code is either combined with the contract implementation
    or stored separately on the blockchain. Several entry points may lead to difficulties.
  \end{enumerate}
\item Implement the transformation in two steps:
  \begin{enumerate}
  \item negation of the formula
  \item followed by the translation of an existentially quantified formula to code that picks a
    random value subject to predicate. For initial experiments, the target of the
    translation can be any programming 
    language; later on, the target has to be a language hosted on the blockchain
    (e.g. Michelson or Solidity). These languages have to be extended with a random
    generator. There has to be a special execution mode for ``challenge contracts''
    that search for a counterexample of an assertion.
  \end{enumerate}
  Negation turns prenex universial quantification into prenex existential quantification. But it
  would be ineffective to generate a random integer and then check that it falls in the
  interesting range (e.g., $2\le n \le \sqrt p$ and $0\le k< n-1$). Rather, the formula
  needs to be analzed and the random generator should be restricted to only generate
  interesting values. 

  The random choice should be performed in a ``clever'' way that does not waste too
  much effort. For a wasteful example, consider~\eqref{eq:2}:
  \begin{gather*}
    (\exists j, k ) (0\le j< k < |a|) \wedge a[j] > a[k]
  \end{gather*}
  In the worst case, we would generate two arbitrary non-negative integers $j$ and $k$
  and then check the predicate. However, the predicate $(0\le j< k < |a|)$ will fail
  almost always, so we do not get to test the interesting part of the predicate
  $a[j]>a[k]$.

  In the best case, the generated testing code would be equivalent to the following code fragment
  in Python
\begin{lstlisting}[language=Python]
def exists_counterexample (a : list) -> bool:
  if len(a) < 2: return False
  j = random.randint (0, len(a) - 2)
  k = random.randint (j+1, len(a) - 1)
  return a[j] > a[k]
\end{lstlisting}

  As another, simpler example consider~\eqref{eq:4} (checking for prime numbers)
\begin{lstlisting}[language=Python]
def exists_counterexample (p : int) -> bool:
  r = math.isqrt (p)
  if r < 2: return False
  n = random.randint (2, r)
  return (p % n) == 0
\end{lstlisting}

  Observe that in each case there is a test to eliminate trivial cases, which avoids
  illegal invocations of \lstinline/randint/.
\item Each test function should come with a formula that gives an estimate of
  its effectiveness, i.e., how many times do we have to run the test to find a
  counterexample with a given probability $p$. Ideally, the formula returns a lower
  bound on the number of tests for any $p$.
\item What is the minimum number $m$ of validators such that the probability of
  spotting an error is less than a given threshold $q$?
\item Instead of using random numbers, is there a way to coordinate a systematic
  exploration of the iteration space? (Is this at all necessary?)
\item What are the connections to random testing? 
\item I think this scheme implements a Bloom filter, globally? We are testing whether a value is a
  member of an implicitly given set. We may accept a value even if it does not belong
  to the set, but if we reject a value, it definitely does not belong to the set.
\end{itemize}

\subsection{Blockchain-Specific Tasks}
\label{sec:blockch-spec-tasks}

\begin{ethereum}
  Ethereum-specific remarks:
  Implementing this scheme requires adding instructions to the virtual machine. One
  instruction would generate a random number (EVM uses 512 bit unsigned integer IIRC),
  the other would raise an overriding failure. This failure will have to indicate that
  validation of the contract fails.
\end{ethereum}

\begin{itemize}
\item The VM interpreter needs to be extended with these instructions.
\item The translation (of formulas) needs to retargetted to VM code. This may have to
  be hacked into the compiler.
\item The translation has to be amended so that counterexamples can also be checked.
\item (extension) Every validator could perform a predetermined finite number of checks for each
  assertion. This would not be significantly more expensive. Moreover, the number of
  checks could be set individually depending on the contract and on the size of the
  input.
\item (extension) One could also check properties of the values stored with the contract.
\end{itemize}
\subsection{Protocol Design}
\label{sec:protocol-design}

\begin{changethis}
  Tezos specific remarks: here the number of validators is really small compared to
  Bitcoin or Ethereum
  \begin{itemize}
  \item Maybe we need to propose and validate the same block multiple times before it
    gets accepted. E.g, a transaction containing this feature is only accepted if it is
    revalidated $n$ times.
  \item Extra validators only for this kind of transaction.
  \item Maybe each validator needs to conduct multiple test runs.
  \item Maybe test runs need to be coordinated
  \end{itemize}
\end{changethis}

\begin{itemize}
\item (Tezos) Proposal for an architecture. The assertion for contract C gets compiled
  to a separate contract A(C). A(C) takes C's argument x and attempts to challenge 
  the assertion by finding a counterexample. Given x, we determine a number n of
  ``probation cycles'' during which the net attempts to find a counterexample. The
  injecting node broadcasts A(C)(x) for n cycles. If A(C)(x) finishes successfully,
  this execution vanishes (as no counterexample has been found). If A(C)(x) finds a
  counterexample, it records the counterexample along with the data to check it on the
  blockchain as in A(C,x,y) where it can be validated by the net as usual.
  After n cycles the injecting node broadcasts C(x), which is processed as
  usual. However, a valid blockchain either contains C(x) or A(C,x,y), but not both. It
  is an error to attempt to enter C(x) after A(C,x,y) is on the chain, and vice versa.

  It needs to be investigated which aspects of the protocol need to be
  adapted. Moreover, A(C,x) must execute in a different mode that enables the random
  instruction. This instruction is not available in the default execution mode.
\item   Once a counterexample has been found,
  this counterexample is broadcast to the network where it can be verified by further
  nodes.  I think this broadcast and the reaction to this broadcast requires an update of
  the protocol: all receivers of the broadcast will check the counterexample and revert
  the transaction. 
\item (see above for a revised approach) A transaction using counterexamples has to
  wait for a couple of cycles in the transaction pool until the node feels safe to
  include the transaction in a block.
  The number of cycles to wait depends on the contract and on the size of the input.
  So that number should be variable.
\item (low priority) This approach poses the danger of a DOS attack. Someone could post a false
  counterexample, which does not really lead to a failure. If everyone switches to
  check this proposed counterexample instead of generating a random number, then the
  real problem may remain hidden. So, there has to be a component in the protocol that
  penalizes someone who broadcasts a false counterexample. Moreover, not all validators
  should switch to recheck a proposed counterexample; or they could check it in
  addition to their own random check.

  More precisely:
  \begin{itemize}
  \item Every validator must run at least one (round of) random tests.  
  \item If it receives a proposed counterexample, it must check the counterexample,
    too.
  \item It remains to investigate a countermeasure for nodes generating a flood of
    false counterexamples, which would keep every validator busy. Perhaps checking of
    the counterexample should also be done probabilistically.
  \end{itemize}
\end{itemize}

\section{Suggestions}

\subsection{General}
\label{sec:suggestions-general}


In the example, we checked if all elements satisfy a certain condition:
\begin{lstlisting}[language=Caml]
forall (n : Int)
   if ( 2 <= n && n <= sqrt(p))
       assert(p%n != 0)
\end{lstlisting}
However, if we consider  parameter p to be a sorted list and we want
to assert that it is indeed sorted, we would need to check two values: 
\begin{lstlisting}[language=Caml]
forall (n : Int, m: Int)
   if ( n > m)
       assert(p.(n) > p.(m))
\end{lstlisting}

You can have foralls with more than one variable, but you could also nest the foralls as in

\begin{lstlisting}[language=Caml]
forall( n : int)
 forall(m : int) ...
\end{lstlisting}
Externally, I think your proposed syntax is nicer, but internally it will be simpler to just have 
\lstinline/Forall( ident, type, expr)/
in the type for the AST.


Somehow the smart random generator will have to know in which range to
generate the values.  Considering the second example, the generator should generate 
\begin{lstlisting}[language=Caml]
m = random(0, length(list))
n = random(m, length(list))
\end{lstlisting}
For this transformation, the assertion should be completely explicit.
That is, in the example, there should be an additional condition

\begin{lstlisting}[language=Caml]
... if (n < length(p) && m < length(p)) ...
\end{lstlisting}
which means we should have the length of a list available as a primitive.


Can we realize this transformation from the above syntax, or do we
need additional syntax that expresses this transformation separately? 
Do I even have to consider these things when creating the grammar?

The process could be something like this:
\begin{enumerate}
\item  calculate the negation of the formula. This amounts to
  swapping forall and exists, skipping the conditions, and negating
  the body of the assertion. This last step should also be done by
  applying de Morgan exhaustively down to the literals. In your
  example you'd end up with

\begin{lstlisting}[language=Caml]
exists(n : int)
 exists(m : int)
    if (n< length(p) && m < length(p))
      if (n>m)
         check( p.(n) <= p.(m) )
\end{lstlisting}
\item look at the atomic constraints that bound the exists and move them upfront. Example: 
  \begin{itemize}
  \item  the first constraint is \lstinline/n<length(p)/ $\to$ you need to break up
    the conjunction
  \item independent of \lstinline/exists(m)/, hence move backward
  \end{itemize}
\begin{lstlisting}[language=Caml]
exists(n : int)
 if (n < length(p))
   exists(m : int)
     if (m < length(p))
       ...
\end{lstlisting}
\item  if the constraint meets its generator (exists), then merge it:

\begin{lstlisting}[language=Caml]
exists(n : int, n < length(p))
\end{lstlisting}
\item  same for m

\begin{lstlisting}[language=Caml]
 exists(m : int, m < length(p))
\end{lstlisting}
( at this point, each generator / exists could come with a number of upper and lower bounds )
\item now we need to consider this

\begin{lstlisting}[language=Caml]
exists(n : int, n < length(p))
 exists(m : int, m < length(p))
   if (n>m)
     ...
\end{lstlisting}
this condition gives us an upper bound on m (this requires some manipulation of the formula, in general)
so we just add it as another bound to m:

\begin{lstlisting}[language=Caml]
exists(n : int, n < length(p))
 exists(m : int, m < length(p) && m < n)
    ...
\end{lstlisting}
\item  you can find out (but this is more involved and thus optional)
  that \lstinline/0 < n/ is required for the \lstinline/exists(m ...)/ to be nonempty and that
  \lstinline/m < n/ implies \lstinline/m < length(p)/: 

\begin{lstlisting}[language=Caml]
exists(n : int, 0 < n,  n < length(p))
 exists(m : int, 0 <=m, m < n)
   check( p.(n) <= p.(m) )
\end{lstlisting}
At this point, you can read off the invocation for the random generator.
If any constraints remain, they'd have to be checked at run time.
In the best case, things work out like in the example. Otherwise, each if-constaint that turns out to be false results in a useless test run.
\end{enumerate}

For background, there is a nice paper about the underlying testing methodology:
How to specify it! by John Hughes:
\url{https://www.tfp2019.org/resources/tfp2019-how-to-specify-it.pdf}


\subsection{Tezos}
\label{sec:suggestions-tezos}


Michelson and the Tezos infrastructure directly support entry points
with a specific notation. Hence, the syntax should reflect and reuse
that notation.

The syntax for assertions should be chosen to be similar to OCaml.
A proposal for the assertion in \eqref{eq:5} could be
\begin{lstlisting}[language=caml]
entrypoint %default
  parameter (_, (_, p)) =
    forall (n : int)
      if (2 <= n && n <= p / 2)
        p % n <> 0
\end{lstlisting}
A parameter phrase may occur multiple times. Its single argument
is a pattern that matches the contract's entrypoint's argument. The type of the
pattern variables (here just \lstinline/p/) is determined from the code of the
contract.

The entrypoint phrase may also occur multiple times. However, it is
illegal to have patterns that overlap, i.e., which match the same
contract invocation. Overlapping may occur due to a repeated pattern
in clauses for the same entrypoint, but also due to a pattern for the
\lstinline/%default/ overlapping with a pattern for another entrypoint.

In Tezos/Michelson, entry points are
modeled by having a sum type as parameter type. For example, a
contract with two entry points can be equipped with assertions as
follows.
\begin{lstlisting}[language=caml]
parameter (Left x) =
  0 < x && x < 100

parameter (Right (a, b)) =
  exists (c : int)
    if (0 < c && c <= a + b)
      a*a + b*b = c*c
\end{lstlisting}
BTW, this is an example for a useful contract with an existential!

These phrases would refer to the \lstinline/%default/ entrypoint, as
none is explicitly specified. Suppose the parameter type for the
contract in question was specified in Michelson as follows.
\begin{lstlisting}
parameter (or (int %set_scale) (pair int int %set_triangle))
\end{lstlisting}
An equivalent way to stating the previous assertions would be to use
entrypoints.
\begin{lstlisting}[language=caml]
entrypoint %set_scale
  parameter (x) =
    0 < x && x < 100

entrypoint %set_triangle
  parameter (a, b) =
    exists (c : int)
      if (0 < c && c <= a + b)
        a*a + b*b = c*c
\end{lstlisting}

The idea would be to have two files with the same basename, but
different extensions. One file contains the Michelson code, the other
the assertion.

\section{Rejected Ideas}
\label{sec:rejected-ideas}

This section contains some ideas that we discussed but rejected. Feel free to resurrect
any of this if you find new arguments.

\begin{itemize}
\item Q: How about instead of generating and checking random number, it could
  store a Bloom filter for already checked number, so a validator can
  generate a random number and check it if it is rejected by Bloom filter,
  but this also happen that there will be some number never get checked
  because it is rejected by Bloom filter, but actually never been checked.

  A: This is not a good idea for several reasons.
  \begin{enumerate}
  \item It's not clear where we should store the Bloom filter. It has to be available
    everywhere, so it has to be stored on the blockchain.
  \item If it's on the blockchain, then it serializes the tests for counterexamples:
    the point of the scheme is to test many random values concurrently without
    interference.
  \item The Bloom filter errs in the wrong direction. It may accept elements that have
    not been added before, which means that elements that should be tested are not. We
    would need a co-Bloom filter that errs in the other direction: it should reject all
    elements that have not been added, and it may also reject elements that have been added.
  \end{enumerate}

\end{itemize}

\end{document}

