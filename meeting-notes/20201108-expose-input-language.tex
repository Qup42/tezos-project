\documentclass{article}

\usepackage[utf8]{inputenc}
\usepackage{amsmath}
\usepackage{mathpartir}
\usepackage{listings}
\usepackage{url}

\usepackage{hyperref}

\usepackage{xcolor}
\usepackage{tcolorbox}
\tcbuselibrary{breakable}

\newenvironment{changethis}{%
  \begin{tcolorbox}[breakable,notitle,boxrule=0pt,colback=blue!20,colframe=blue!20]}{%
  \end{tcolorbox}}

\newenvironment{ethereum}{%
  \begin{tcolorbox}[breakable,notitle,boxrule=0pt,colback=red!20,colframe=red!20]}{%
  \end{tcolorbox}}


% Copyright 2017 Sergei Tikhomirov, MIT License
% https://github.com/s-tikhomirov/solidity-latex-highlighting/

\usepackage{listings, xcolor}

\definecolor{verylightgray}{rgb}{.97,.97,.97}

\lstdefinelanguage{Solidity}{
	keywords=[1]{anonymous, assembly, assert, balance, break, call, callcode, case, catch, class, constant, continue, constructor, contract, debugger, default, delegatecall, delete, do, else, emit, event, experimental, export, external, false, finally, for, function, gas, if, implements, import, in, indexed, instanceof, interface, internal, is, length, library, log0, log1, log2, log3, log4, memory, modifier, new, payable, pragma, private, protected, public, pure, push, require, return, returns, revert, selfdestruct, send, solidity, storage, struct, suicide, super, switch, then, this, throw, transfer, true, try, typeof, using, value, view, while, with, addmod, ecrecover, keccak256, mulmod, ripemd160, sha256, sha3}, % generic keywords including crypto operations
	keywordstyle=[1]\color{blue}\bfseries,
	keywords=[2]{address, bool, byte, bytes, bytes1, bytes2, bytes3, bytes4, bytes5, bytes6, bytes7, bytes8, bytes9, bytes10, bytes11, bytes12, bytes13, bytes14, bytes15, bytes16, bytes17, bytes18, bytes19, bytes20, bytes21, bytes22, bytes23, bytes24, bytes25, bytes26, bytes27, bytes28, bytes29, bytes30, bytes31, bytes32, enum, int, int8, int16, int24, int32, int40, int48, int56, int64, int72, int80, int88, int96, int104, int112, int120, int128, int136, int144, int152, int160, int168, int176, int184, int192, int200, int208, int216, int224, int232, int240, int248, int256, mapping, string, uint, uint8, uint16, uint24, uint32, uint40, uint48, uint56, uint64, uint72, uint80, uint88, uint96, uint104, uint112, uint120, uint128, uint136, uint144, uint152, uint160, uint168, uint176, uint184, uint192, uint200, uint208, uint216, uint224, uint232, uint240, uint248, uint256, var, void, ether, finney, szabo, wei, days, hours, minutes, seconds, weeks, years},	% types; money and time units
	keywordstyle=[2]\color{teal}\bfseries,
	keywords=[3]{block, blockhash, coinbase, difficulty, gaslimit, number, timestamp, msg, data, gas, sender, sig, value, now, tx, gasprice, origin},	% environment variables
	keywordstyle=[3]\color{violet}\bfseries,
	identifierstyle=\color{black},
	sensitive=false,
	comment=[l]{//},
	morecomment=[s]{/*}{*/},
	commentstyle=\color{gray}\ttfamily,
	stringstyle=\color{red}\ttfamily,
	morestring=[b]',
	morestring=[b]"
}

\lstset{
	language=Solidity,
	backgroundcolor=\color{verylightgray},
	extendedchars=true,
	basicstyle=\footnotesize\ttfamily,
	showstringspaces=false,
	showspaces=false,
	numbers=left,
	numberstyle=\footnotesize,
	numbersep=9pt,
	tabsize=2,
	breaklines=true,
	showtabs=false,
	captionpos=b
}


\lstset{numbers=none}

\title{Expos\'{e}\\
  Input Language for Smart Contract API}
\author{Peter Thiemann}
\begin{document}
\maketitle{}

\section{Introduction}
\label{sec:introduction}

The current Tezos API for invoking contracts with parameters is
untyped. This manifests in the input of the Tezos client where
parameters have to be supplied as strings and are then parsed before
they are passed to the contract execution. For example:
\begin{lstlisting}[language=sh]
tezos-client transfer 10000 from contractOwner to v1-public-pool 
--arg '(Left Unit)' --burn-cap 0.004
\end{lstlisting}
This is also true for the first cut of the OCaml API.

 Here we propose a DSL to make it typed.
First we recall some information from the Michelson white paper
\url{https://tezos.gitlab.io/whitedoc/michelson.html#core-data-types-and-notations}. 

\subsection{Core data types and notations}

\begin{itemize}
\item \texttt{string}, \texttt{nat}, \texttt{int} and \texttt{bytes}: The core primitive constant types.
\item 
  \texttt{bool}: The type for booleans whose values are \texttt{True} and \texttt{False}.
\item 
  \texttt{unit}: The type whose only value is \texttt{Unit}, to use as a placeholder
  when some result or parameter is not necessary. For instance, when
  the only goal of a contract is to update its storage.
\item 
  \texttt{list (t)}: A single, immutable, homogeneous linked list, whose
  elements are of type (t), and that we write \texttt{\{\}} for the empty list or
  \texttt{\{ first ; ... \}}. 
\item 
  \texttt{pair (l) (r)}: A pair of values a and b of types (l) and (r), that we
  write \texttt{(Pair a b)}.
\item 
  \texttt{option (t)}: Optional value of type (t) that we write \texttt{None} or \texttt{(Some
  v)}.
\item 
  \texttt{or (l) (r)}: A union of two types: a value holding either a value a
  of type (l) or a value b of type (r), that we write \texttt{(Left a)} or
  \texttt{(Right b)}.
\item 
  \texttt{set (t)}: Immutable sets of values of type (t) that we write as lists
  \texttt{\{ item ; ... \}}, of course with their elements unique, and sorted.
\item 
  \texttt{map (k) (t)}: Immutable maps from keys of type (k) of values of type
  (t) that we write \texttt{\{ Elt key value ; ... \}}, with keys sorted.
\item 
  \texttt{big\_map (k) (t)}: Lazily deserialized maps from keys of type (k) of
  values of type (t) that we write \texttt{\{ Elt key value ; ... \}}, with keys
  sorted. These maps should be used if one intends to store large
  amounts of data in a map. They have higher gas costs than standard
  maps as data is lazily deserialized. A \texttt{big\_map} cannot appear inside
  another \texttt{big\_map}.
\end{itemize}

\subsection{Relation to OCaml datatypes}
\label{sec:relat-ocaml-datatyp}

\begin{center}
  \begin{tabular}{|l|l|}
    Michelson & OCaml \\\hline
    \texttt{string}& \texttt{string}  \\
    \texttt{nat} & ? (\texttt{int}) \\
    \texttt{int} & \texttt{int} \\ 
    \texttt{bytes}& ? \\
    \texttt{bool}&\texttt{bool} \\
    \texttt{unit}&\texttt{unit} \\
    \texttt{list t}&\texttt{t list} \\
    \texttt{pair l r}&\texttt{l * r} \\
    \texttt{option t}     &\texttt{t option} \\
    \texttt{or l r}&\texttt{[`Left of l | `Right of r]} \\
    \texttt{set t}&? \texttt{Set.Make (T).t} \\
    \texttt{map k t}&? \texttt{`t Map.Make(K).t} \\
    \texttt{big\_map k t}&? \texttt{`t Map.Make(K).t} \\
    \hline
  \end{tabular}
\end{center}

\section{Tasks}
\label{sec:tasks}

We can represent Michelson parameter values with the following OCaml type using GADTs.
\begin{lstlisting}[language=Caml]
type _ Mvalue =
  | Mstring : string -> string Mvalue
  | Mint  : int -> int Mvalue
  | Mbool : bool -> bool Mvalue
  | Munit : unit -> unit Mvalue
  | Mnil  : 't list Mvalue
  | Mcons : 't Mvalue * 't list Mvalue -> 't list Mvalue
  | Mpair : 'l Mvalue * 'r Mvalue -> ('l * 'r) Mvalue
  | Mnone : 't option Mvalue
  | Msome : 't Mvalue -> 't option Mvalue
  | Mleft : 'l Mvalue -> [`Left of 'l | `Right 'r] Mvalue
  | Mright : 'r Mvalue -> [`Left of 'l | `Right 'r] Mvalue
\end{lstlisting}

\begin{lstlisting}[language=Caml]
type _ Mtype = 
  | Tstring : string Mtype
  | Tint    : int Mtype
  | Tbool   : bool Mtype
  | Tunit   : unit Mtype
  | Tlist   : 't Mtype -> 't list Mtype
  | Tpair   : 'l Mtype * 'r Mtype -> ('l * 'r) Mtype
  | Toption : 't Mtype -> 't option Mtype
  | Tor     : 'l Mtype * 'r Mtype 
            -> [`Left of 'l | `Right of 'r] Mtype
\end{lstlisting}
Instead of using polymorphic variants, one could also define a binary
sum type like Haskell's Either type.
\begin{lstlisting}[language=Caml]
type ('a, 'b) either = Left of 'a | Right of 'b
\end{lstlisting}

To read input values, we'd need a function like the following.
\begin{lstlisting}[language=Caml]
mread : 't Mtype -> 't -> 't Mvalue
\end{lstlisting}

Or given a contract for index type \lstinline/t/, we'd have an
invocation function
\begin{lstlisting}[language=Caml]
invoke : 't Mtype Tezos.contract -> 't -> result Tezos.apimonad
\end{lstlisting}
with an internal serialization function
\begin{lstlisting}[language=Caml]
serialize : 't Mtype -> 't -> string
\end{lstlisting}
Not sure if deserialization is also needed, but it might be necessary
to decode the current state of a contract. Deserialization may fail
and thus raise an exception.
\begin{lstlisting}[language=Caml]
deserialize : 't Mtype -> string -> 't
\end{lstlisting}

One can get rid of the \lstinline/Mtype/ datatype by directly
composing serializers in tagless style. I think. But having the type
representation might be more flexible.
\begin{lstlisting}[language=Caml]
s_string : string -> string
s_int : int -> string
s_bool : bool -> string
s_unit : unit -> string
s_list : ('t -> string) -> ('t list -> string)
s_pair : ('l -> string) -> ('r -> string) -> ('l * 'r -> string)
s_option : ('t -> string) -> ('t option -> string)
s_or : ('l -> string) -> ('r -> string) 
     -> ([`Left of 'l | `Right of 'r] -> string)
\end{lstlisting}

Example
\begin{lstlisting}[language=Caml]
mytype = Tor (Tbool, Toption (Tstring))
myserializer = serialize mytype : 
               [`Left of bool | `Right of string option] -> string

tlserializer = s_or s_bool (s_option s_string) :
               [`Left of bool | `Right of string option] -> string
\end{lstlisting}

Query the Tezos blockchain for the type of a contract and its
entrypoints.
\begin{lstlisting}[language=Caml]
make_typed_contract : contract_hash -> 't Mtype -> 't Contract
\end{lstlisting}
The function call \lstinline/make_typed_contract hash mty/ would query
the node (via client lib or rpc) for the type of the contract
\lstinline/hash/ and check it against \lstinline/mty/.
This call fails if the type \lstinline/mty/ does not match the
stored type of the contract.
The returned value contains the serializer from \lstinline/'t -> string/.
\begin{lstlisting}[language=Caml]
invoke_typed : 't Contract -> 't -> result Tezos.apimonad
\end{lstlisting}
\end{document}

