\documentclass{article}

\usepackage[utf8]{inputenc}
\usepackage{amsmath}
\usepackage{mathpartir}
\usepackage{listings}
\usepackage{url}
\usepackage{xcolor}
\usepackage{tcolorbox}

\title{Expos\'{e}\\
  Proof of Vaccination on the Blockchain}
\author{Peter Thiemann}

\begin{document}
\maketitle

\section{Introduction}
\label{sec:introduction}

The world is racing towards containing the Corona virus and fighting the pandemic.
One important step in this fight is to vaccinate as many people as possible.
At some point, it is expected that certain activities (e.g., air travel, cruises, sports) are open
exclusively to vaccinated  persons. To this end, there must be a central registry for all
vaccinations so that everyone can prove their vaccination status.

Trust is a major issue in this registry. Only certified authorities should be able to enter
vaccination records. It should not be possible to enter fake records. Each record should be linked
to an individual but without subverting data protection and privacy. Everyone
should be able to verify a vaccination given the identity of an individual within seconds.
A certified authority should be able to obtain the full record (which vaccine, when vaccinated)
within a reasonable time.

Entering records or obtaining a full record may incur a reasonable cost. Verifying a vaccination
should be as cheap as possible.

Authorities should be structured hierarchically. For example, there is a small number of worldwide
authorities; they vouch for national authorities, which in turn vouch for regional authorities and
then local authorities (e.g., doctors or vaccination centers). Such a structure is the standard for
certificates for webpages, emails, etc.

The goal of the thesis is to investigate whether the blockchain is a suitable technical foundation
for building such a registry. At first glance, this seems to be the case because the blockchain is a
global decentralized database that can be maintained by mutually untrusting communities.

The main concern is the cost to store the (huge) information on the blockchain and the cost and time
to verify.  So the most important task is to find out how to store and then verify the
vaccinations records with reasonable cost and time.

\section{Outline}
\label{sec:outline}

The thesis touches on several issues in computer science and economics.

\begin{enumerate}
\item Software engineering:
  \begin{itemize}
  \item identify the actors, design a data model, and specify the
    actions. Keep the data model simple, but realistic. Apply
    domain-driven engineering if possible.
  \item The data model maintains of persons to \emph{records} of vaccination.
  \item identify the non-functional requirements: size of the personal
    certificate, time needed to create the certificate (do you get it
    when you leave the doctor, or is it sufficient to get it the next day)
  \end{itemize}

\item Economics and scalability:
  \begin{itemize}
  \item what is the cost for registering a vaccination? Some cost is
    acceptable, but it should be kept down as much as possible.
  \item what is the cost for checking if someone if vaccinated? This
    check should be cheap and it should take no longer than a few
    seconds, even for millions of vaccination records
  \item consider the datastructures required. what is the best way of
    mapping the data to the blockchain?
  \end{itemize}
\item Programming:
  \begin{itemize}
  \item build a prototype implementation consisting of smart contracts
    and web-based user interfaces.
  \item some authorization infrastructure is necessary for
    registering; this is not the key focus, so some PK infrastructure
    for digital signatures should be sufficient.
  \item Records must be verifiably personalized.
  \item  Full records may not be needed, for example, it might be
    possible to collect many records in a Merkle tree and only store
    the hash on the blockchain. The certificate would contain a proof
    of containment of the record in the Merkle tree.
  \item optional: which blockchain platform is most suitable? Ethereum
    seems like the standard. Tezos? Permissioned blockchains? See here
    for an overview: \url{https://www.sciencedirect.com/science/article/pii/S2405959520301909}
  \end{itemize}
\end{enumerate}

\end{document}
